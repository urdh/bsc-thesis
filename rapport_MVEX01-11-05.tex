\documentclass[a4wide,twoside,notitlepage,openright]{report}
\usepackage[utf8]{inputenc}

\usepackage[T1]{fontenc}
\usepackage{lmodern}

\usepackage[british,swedish]{babel}
\usepackage[pdftex]{graphicx}
\usepackage{ifpdf}
\ifpdf\DeclareGraphicsExtensions{.pdf,.png,.jpg}\fi
\usepackage{epstopdf}
\usepackage{amsmath, amssymb}
\usepackage{epsfig}
%\usepackage{floatflt}
%\usepackage{ae}
\usepackage{icomma}
%\usepackage{subfig}
\usepackage{subfiles}
\usepackage{fancyhdr}
\usepackage{hyperref}
\urlstyle{same}
\usepackage{moreverb}
\usepackage{textcomp}
\usepackage{harvard}
\usepackage[usenames,dvipsnames]{color}
\usepackage{fancyhdr}
\usepackage{paralist}
\usepackage{parskip}
\usepackage{booktabs}
\usepackage{xspace}
\usepackage[font=small,format=plain,labelfont=bf,textfont=it]{caption,subfig}
%\usepackage{afterpage}
\usepackage{slashbox}
\usepackage{tikz}
\usepackage[noauto]{chappg}
\usepackage[swedish,refpages]{gloss}
\usepackage[swedish]{varioref}

% För att kunna kommentera bort varioref om det inte är bra
\ifdefined\vref\relax\else\newcommand{\vref}[1]{\ref{#1}}\fi

%% Minted-saker. Aktivera med \mintedtrue, avaktivera med \mintedfalse
\newif\ifminted
\mintedfalse % <-- denna raden (av)aktiverar minted
\ifminted
	\usepackage{minted}
	\newminted{matlab}{linenos,mathescape,bgcolor=codebg}
	\usemintedstyle{manni}
	\renewcommand{\listingscaption}{Kodfragment}
\fi

% Tikz-flowcharts
\usetikzlibrary{shapes,arrows}
\tikzstyle{decision} = [diamond, draw, fill=black!5, 
    text width=6em, text badly centered, node distance=3cm, inner sep=0pt]
\tikzstyle{block} = [rectangle, draw, fill=black!5, 
    text width=10em, text centered, rounded corners, minimum height=4em]
\tikzstyle{line} = [draw, -latex']

% Titel här istället för topmatter.tex för att kunna använda i headers
\newcommand{\titel}{Statistisk bildanalys av handgester för människa-dator-interaktion} %Skriv in titeln här
\newcommand{\undertitel}{} %Skriv in undertiteln här, eller lämna tomt

% Fina inställningar
\frenchspacing
\citationmode{abbr}
\setcounter{secnumdepth}{3} % Osäker på dessa två..
\setcounter{tocdepth}{3}    % Vad blir snyggast - med eller utan?
\definecolor{light-gray}{gray}{0.95}
\let\oldparagraph=\paragraph
\let\oldsubparagraph=\subparagraph
\renewcommand{\paragraph}[1]{\oldparagraph[#1]{#1.}}
\renewcommand{\subparagraph}[1]{\oldsubparagraph[#1]{\normalfont\itshape#1.}}
\renewcommand{\thefootnote}{\fnsymbol{footnote}}
\newcommand{\notes}[1]{{\color{BrickRed}\textbf{Anteckning: }\color{Red}#1}}
\renewcommand{\subsectionmark}[1]{\markright{\thesubsection~\textit{#1}}{}}
\fancypagestyle{plain}{
  \fancyhf{} % remove everything
  \fancyhead[LE,RO]{\thepage}  
  \renewcommand{\headrulewidth}{0.4pt}
  \renewcommand{\footrulewidth}{0pt}
}
\fancypagestyle{fancier}{
  \fancyhf{}
  \fancyhead[LE,RO]{\thepage}
  \fancyhead[RE]{\textsc{\titel}}
  \fancyhead[LO]{\rightmark}
  \renewcommand{\headrulewidth}{0.4pt}
  \renewcommand{\footrulewidth}{0pt}
}
%\afterpage{\clearpage}
\let\oldquote=\quote
\expandafter\def\expandafter\quote\expandafter{\quote\small}
\definecolor{codebg}{rgb}{0.975,0.975,0.975}
% Abstract-hack
\makeatletter
\renewenvironment{abstract}{\if@twocolumn\section*{\abstractname}
\else\small\begin{center}{\bfseries \abstractname\vspace{-.5em}\vspace{\z@}}
\end{center}\oldquote\fi}{\if@twocolumn\else\endquote\fi}
\makeatother

% Kommandon
\newcommand{\vect}[1]{\ensuremath{\mathbf{#1}}}  % Vektorer
\newcommand{\N}{\ensuremath{\mathcal{N}}}     % Normalfördelning
\newcommand{\Prob}{\ensuremath{\mathrm{P}}}   % Sannolikhet av
\newcommand{\R}{\ensuremath{\mathbb{R}}}      % Reella tal
\newcommand{\rd}{\ensuremath{\mathrm{d}}}     % 'rakt' d
\newcommand{\id}{\ensuremath{\,\rd}}          % Integral-d
%\newcommand{\MATLAB}{MAT\-LAB\textregistered\xspace}         % MATLAB
\newcommand{\MATLAB}{MAT\-LAB\xspace}         % MATLAB
\newcommand{\knn}{$k$NN\xspace}              % kNN

% Glossary
\makegloss
\gloss[nocite]{*}

\begin{document}

% Titelsida
\subfile{topmatter.tex}

% Förord
\pagenumbering{Roman}
\pagestyle{plain}
%\chapter*{\vspace{-2.5em} Förord\vspace{-1em}}
\chapter*{Förord}
\subfile{kapitel/forord.tex}
\cleardoublepage
\tableofcontents
\cleardoublepage
\printgloss{glossary}
\cleardoublepage
\pagestyle{fancier}

% Inledning
\pagenumbering{arabic}
\chapter{Inledning}\label{sec:inledning}
\subfile{kapitel/inledning.tex}
\subfile{kapitel/problem.tex}
\subfile{kapitel/avgransning.tex}

\cleardoublepage

% Huvuddel
% Ändra inbördes ordning på allt det här så att det blir logiskt.
% Ta även bort kapitel som vi inte ska ha, och lägg till nya.
\chapter{Teori}
\subfile{kapitel/teori.tex}
\subfile{kapitel/klassificering.tex}
\subfile{kapitel/hudklassificering.tex}
\subfile{kapitel/features.tex}
\subfile{kapitel/knn.tex}
\subfile{kapitel/HMM.tex}
\cleardoublepage

\chapter{Metod}
\subfile{kapitel/metod.tex}
\subfile{kapitel/metod_gester.tex}
\subfile{kapitel/metod_hud.tex}
\subfile{kapitel/metod_datainsamling.tex}
\subfile{kapitel/metod_knn.tex}
\subfile{kapitel/metod_hmm.tex}
\cleardoublepage

\chapter{Resultat}
\subfile{kapitel/resultat.tex}
\subfile{kapitel/resultat_hudklassificering.tex}
\subfile{kapitel/resultat_knn.tex}
\cleardoublepage

\chapter{Diskussion}
\subfile{kapitel/diskussion.tex}
%\subfile{kapitel/diskussion_xxx.tex}
%\subfile{kapitel/diskussion_yyy.tex}

\cleardoublepage

% Källförteckning
\pagestyle{plain}
\bibliographystyle{dcumod}
\bibliography{referenser}

% Bilagor
\appendix
\let\oldchapter=\chapter
\renewcommand{\chapter}[1]{\oldchapter{#1}\pagenumbering{bychapter}}
\cleardoublepage
\pagestyle{fancier}
\renewcommand{\rightmark}[1]{\appendixname\ \thechapter}

\chapter{\MATLAB-kod}\label{sec:matlab}
\ifminted\subfile{kapitel/appendix-matlab.tex}\fi
\chapter{Fullständigt resultat för egenskaper}\label{sec:egenskapsresultat}
\subfile{kapitel/app_res_egenskaper.tex}

\end{document}

% Chalmers referensguide rekommenderar Harvard-stil (författare/årtal):
%  http://www.lib.chalmers.se/education/guides/references/

% Följande kommandon finns tillgängliga för att referera:
% \citeasnoun{..}    ger   "Författare (2000)"
% \cite{..}          ger   "(Författare, 2000)"
% \possessivecite{..} ger   "Författares (2000)"
% \citeaffixed{..}{t.ex.} ger "(t.ex. Författare, 2000)"
%
% Man kan även referera till specifika sidor: \cite[s.~32]{..}
