\documentclass[a4wide]{article}
\usepackage{graphicx}
\usepackage{amsmath, amssymb}
\usepackage{epsfig}
\usepackage{floatflt} %för inkapslade bilder.
\usepackage{fancyhdr}
\addtolength{\textwidth}{20mm}
%\addtolength{\textheight}{30mm}
\addtolength{\textheight}{10mm}
\addtolength{\headheight}{-10mm}
\addtolength{\oddsidemargin}{-10mm}

% Här nedan kommer kommandon där du skall göra val
%


\newcommand{\ARBETE}{ %välj arbetsbenämning här, notera att båda kan användas samtidigt
 \newline \noindent Examensarbete för kandidatexamen i matematik vid Göteborgs universitet % om någon i gruppen läser på GU
 \medskip
 \newline \noindent Kandidatarbete inom civilingenjörsutbildningen vid Chalmers % om någon läser på Chalmers
 \medskip
}


\newcommand{\titel}{Dokumentnamn} %Skriv in titeln här
\newcommand{\undertitel}{undertitel  } %Skriv in undertiteln här, eller lämna tomt


\newcommand{\namn}{ %Skriv in medlemmarnas namn i bokstavsordning.
  förnamn efternamn 1 \\
  förnamn efternamn 2  \\
%  förnamn efternamn 3 \\
%  förnamn efternamn 4
%  förnamn efternamn 5 \\
%  förnamn efternamn 6 %
}

\newcommand{\examina}{ % här skall gruppens medlemmar skrivas in vid önskad examen. Aktivera aktuella examina, inte kursnummer. Vid fyra med samma examen blir det snyggast om man gör en tabell.
%  \newline \noindent \begin{tabular}{ll}
%  förnamn efternamn 1 &
%  förnamn efternamn 2  \\
%  förnamn efternamn 3 &
%  förnamn efternamn 4
% \end{tabular}
% Vid fem eller sex görs tabellen med tre kolumner
% Vid flera examina skall \bigskip aktiveras utom för den sista.
%
%
%%%%%%%%%%%%%%%%%% Kurs MMG900 %%%%%%%%%%%%%%%%
 \newline \noindent {\it Examensarbete för kandidatexamen i matematik vid Göteborgs universitet} \smallskip
 \newline \noindent förnamn efternamn 1
% \quad förnamn efternamn 2
% \quad förnamn efternamn 3
% \quad förnamn efternamn 4
 \bigskip
%%%%%%%%%%%%%%%%%% Kurs MMG910  %%%%%%%%%%%%%%%%
 \newline \noindent{ \it  Examensarbete för kandidatexamen i matematik inom matematikprogrammet vid Göteborgs universitet} \smallskip
 \newline \noindent förnamn efternamn 1
% \quad förnamn efternamn 2
% \quad förnamn efternamn 3
% \quad förnamn efternamn 4
 \bigskip
 %
%%%%%%%%%%%%%%%%%% Kurs MMG920  %%%%%%%%%%%%%%%%%
%  \newline \noindent {\it Examensarbete för kandidatexamen i tillämpad matematik inom matematikprogrammet vid Göteborgs universitet} \smallskip
%  \newline \noindent förnamn efternamn 1
%  \quad förnamn efternamn 2
%  \quad förnamn efternamn 3
%  \quad förnamn efternamn 4
%  \bigskip
%
%%%%%%%%%%%%%%%%%%% Kurs MSG900  %%%%%%%%%%%%%%%%%
% \newline \noindent {\it Examensarbete för kandidatexamen i matematisk statistik vid Göteborgs universitet} \smallskip
% \newline \noindent förnamn efternamn 1
% \quad förnamn efternamn 2
% \quad förnamn efternamn 3
% \quad förnamn efternamn 4
% \bigskip
%
%%%%%%%%%%%%%%%%%%% Kurs M1G910  %%%%%%%%%%%%%%%%%
%\newline \noindent {\it Examensarbete för kandidatexamen i matematisk statistik inom matematikprogrammet vid Göteborgs universitet} \smallskip
%
%\newline \noindent förnamn efternamn 1
%\quad förnamn efternamn 2
%\quad förnamn efternamn 3
%\quad förnamn efternamn 4
% \bigskip
%
%%%%%%%%%%%%%%%%%%% Kurs MVEX01  %%%%%%%%%%%%%%%%%
 \newline \noindent {\it Kandidatarbete i matematik inom civilingenjörsprogrammet Automation och mekatronik vid Chalmers} \smallskip
 \newline \noindent förnamn efternamn 1
% \quad förnamn efternamn 2
% \quad förnamn efternamn 3
% \quad förnamn efternamn 4
 % \bigskip
%
% \newline \noindent {\it Kandidatarbete i matematik inom civilingenjörsprogrammet Datateknik vid Chalmers} \smallskip
% \newline \noindent förnamn efternamn 1
% \quad förnamn efternamn 2
% \quad förnamn efternamn 3
% \quad förnamn efternamn 4
% \bigskip
%
% \newline \noindent {\it Kandidatarbete i matematik inom civilingenjörsprogrammet Maskinteknik vid Chalmers} \smallskip
%
% \newline \noindent förnamn efternamn 1
% \quad förnamn efternamn 2
% \quad förnamn efternamn 3
% \quad förnamn efternamn 4
% \bigskip
%
%   \newline \noindent {\it Kandidatarbete i matematik inom civilingenjörsprogrammet Teknisk fysik vid Chalmers} \smallskip
%   \newline \noindent förnamn efternamn 1
%   \quad förnamn efternamn 2
%   \quad förnamn efternamn 3
%   \quad förnamn efternamn 4
%   \bigskip
%
%   \newline \noindent {\it Kandidatarbete i matematik inom civilingenjörsprogrammet Teknisk matematik vid Chalmers} \smallskip
%   \newline \noindent förnamn efternamn 1
%   \quad förnamn efternamn 2
%   \quad förnamn efternamn 3
%   \quad förnamn efternamn 4
%   \bigskip
%
 \bigskip

}


\newcommand{\handledare}{% om samtliga handledare är från MV lämnas fältet inst tomt
namn 1& inst \\
%  & namn 2 & inst \\
%  & namn 3 & inst \\
}

\newcommand{\examinator}{
namn %Skriv in examinators namn här.
}

% Här nedan kommer kommandon som du inte skall ändra, de ger utformningen av rapporten. Vid problem kontakta C-H.

\newcommand{\skribenter} {\begin{tabular}{l} \namn \end{tabular}}


%%%%%%%%%%%%%%%%% Här börjar utformningen av omslaget %%%%%%%%%%%%%%%%
\newcommand{\omslag}{
\addtolength{\textheight}{-20mm}

\thispagestyle{fancy}
\newdimen\headrulewidth

\hspace*{-10ex}\parbox{17cm}{
\vspace*{-15mm}
\includegraphics[width=160mm]{LogoCH_GU_SVEsvart.eps}

\vspace{60mm}

\noindent{\Huge \titel}
\bigskip

\noindent {\huge \undertitel}

\noindent\hspace*{-1 ex}{\Large \it
\ARBETE
}
\vspace{20mm}

\noindent\hspace*{-1 ex}\parbox{80mm}{\noindent {\huge \skribenter}}}



\lfoot{\hspace{-6.4ex}\parbox{8cm}{\noindent\large{Institutionen för matematiska vetenskaper\\
Chalmers tekniska högskola\\
Göteborgs universitet\\
Göteborg 2011} }  }
\cfoot{}
\rfoot{}
\lhead{}
\chead{}
\rhead{}
\newpage
\thispagestyle{empty}
\mbox{}

\newpage}
%%%%%%%%%%%%%% Utformning av omslag slut %%%%%%%%%%%%%%%%%%%%

%%%%%%%%%%%%%%%%%%%% Här börjar utformning av titelsidor %%%%%%%%%%%%%%%%
\newcommand{\titelsidor}{

\thispagestyle{fancyplain}


\lfoot{}
\rfoot{}
\lhead{}
\chead{}
\rhead{}




\lfoot{%\hspace*{40mm}
\parbox{70mm}{
Institutionen för matematiska vetenskaper\\
Chalmers tekniska högskola\\
Göteborgs universitet\\
Göteborg 2011
}
}




%\thispagestyle{empty}
\addtolength{\textheight}{-40mm}
%\begin{center}
\addtolength{\topmargin}{30mm}
\noindent {\LARGE \titel}\bigskip\bigskip

\noindent {\large \undertitel}
\vspace{30mm}

%\end{center}

\noindent {\large \examina}
\vfill

\hspace{-5.8 ex} \begin{tabular}[t]{lll}
Handledare:& \handledare
Examinator:& \examinator & \end{tabular}
\bigskip\bigskip


 \newpage
 \thispagestyle{empty}\mbox{}

\vfill

%\begin{center}\hrule  Institutionen för matematiska vetenskaper \\ Göteborg 2009 \end{center}
 \setcounter{page}{0}
\newpage }



\usepackage[T1]{fontenc}                % För svenska bokstäver
\usepackage[swedish,english]{babel}             % För svensk avstavning och svenska rubriker (t ex "innehållsförteckning)


\begin{document}
\selectlanguage{swedish}
\omslag

\titelsidor
\thispagestyle{empty}
\setlength{\textheight}{240mm}
\addtolength{\topmargin}{-50mm}

\begin{abstract}

\end{abstract}

\selectlanguage{english}
\begin{abstract}

\end{abstract}

\newpage
\selectlanguage{swedish}
\pagestyle{plain}
\tableofcontents                % Innehållsförteckning

                % Tabellförteckning
 \newpage
\section*{Förord}
\documentclass[../rapport_MVEX01-11-05]{subfiles}
\begin{document}


    \subsection*{Tack till}
    Mats Rudemo och Magnus Röding för stöd och tips i arbetet
    och skrivprocessen. Många tack även till Hans
    Malmström för värdefulla synpunkter på den språkliga utformningen.
    Tack till våra tålmodiga försökspersoner Emma Kjelsson, Anna Nilsson,
    Jonatan Rydberg, Dundar Göc, Susanne Schilliger Kildal, Gustav Hansson,
    Fredrik Johnsson, Josefin Lövmark, Jakob Friman och Christoffer Johansson
    för era hjälpande händer. Tack till stöttande vänner och givmilda
    vädergudar. Slutligen ett stort tack till versionshanteringstjänsterna Bitbucket och Mercurial.

    \subsection*{Bildrättigheter}
    Figur \ref{fig:knn-overview}: \copyright Antti Ajanki, 2007.\\

    
I denna skall det anges vilka delar som skall tillskrivas respektive
författare. Där skall också anges att en loggbok förts över de
enskilda medverkandes prestationer under arbetet.
\marginpar{+ kommentar om att vi kanske har vävt ihop våra bitar med varandra
sedan\ldots (vilket vi bör göra snyggt)}
Enligt reglerna får examination av examensarbete med flera författare
inte ske utan att loggbok\footnote{Med loggbok menas här gruppens
dagbok och sammanställning av de individuella tidsloggarna} inlämnats
och rapportens inledning uppfyller villkoren ovan.

\subsection*{Prestationsredovisning}
Vi har bla bla bla

Harald Freij

Viktor Nilsson har skrivit den ursprunliga texten i följande kapitel:
\ref{sec:knn}, \ref{sec:inledning}

David Samuelsson

Simon Sigurdhsson \ref{sec:features}, \ref{sec:resultat_features},
\ref{sec:inledning}, \ref{sec:klassificering:fargrymd},
\ref{sec:klassificering:ycbcr}, \ref{sec:klassificering:morfologi}, \ref{sec:klassificering}

\end{document} 
%TODO: mer todo


\newpage
\section{Inledning}
\documentclass[../rapport_MVEX01-11-05]{subfiles}
\begin{document}
Moderna och innovativa gränssnitt för interaktion med datorer har länge varit
en vision inom filmindustrin. Redan på mitten av nittiotalet hänfördes
tittare av 3D-gränssnitt i filmen Hackers, och i början av 2000-talet visade
scener i Minority Report hur Tom Cruise styr ett operativsystem endast med
sina händer med hjälp av två trådlösa handskar.

Denna framtidsvision kan
ligga mycket närmare än vad Spielberg förväntat sig. Redan i början av
nittiotalet gjordes försök att tolka mänskliga rörelser i bildsekvenser
\cite{Yamato92}, och sedan dess har intresset i forskarvärlden bara ökat.
Handgester för kommunikation med datorer är alltså inte längre något som
hör hemma i filmens värld, utan kan snart vara verklighet.

Det finns två sätt att uppnå denna vision; den för användaren mer krävande
metoden med sensorfyllda handskar och den mer lättillgängliga kamerabaserade
metoden som använder data från exempelvis vanliga webbkameror. Den senare har
fördelen att gemene man kan utnyttja lösningen utan några extra hårdvarukrav.

För det kamerabaserade tillvägagångssättet är det mest grundläggande problemet
att över huvud taget kunna identifiera en människas hud i bilder, och det
problemet har många lösningar. Metoderna förfinas hela tiden, men i stort har
det inte hänt något revolutionerande sedan början av 2000-talet
\cite{Sebe04,Kruppa02,Albiol01,Brand00}, med några få undantag
\citeaffixed{Hassanpour08,Khan10}{t.ex.}. När man väl löst detta problem måste datorn
även kunna tolka handens form och rörelser. Detta är ett svårare problem, men
har också behandlats flitigt \cite{Pavlovic97,Garg09,Nielsen04,Zabulis09}.

%\notes{Från \texttt{HMM.tex}:
%
%HMM tillämpat på gestigenkänning och det
%relaterade problemet röstigenkänning är inget nytt. Redan i
%slutet av 60-talet presenterade Baum \cite{Baum66,Baum67,Baum68,Baum70,Baum72} den
%grundläggande teorin, och i början av 70-talet
%implementerades modellen i samband med röstigenkänning av både
%\citeasnoun{Baker75} och \citeasnoun{Jelinek69}. I slutet av
%80-talet presenterades en sammanfattande artikel om röstigenkänning
%med hjälp av HMM \cite{Rabiner89}, som till stor del ligger till
%grund för följande teoretiska framställning. Tillämpningar gentemot
%gestigenkänning gjordes bland annat i början av 90-talet av \citeasnoun{Yamato92},
%och så sent som 2008 av \citeasnoun{Elmezain08}.}

Trots denna uppmärksamhet från forskarvärden och på senare tid även den kommersiella sektorn
 (t.ex.~Microsoft Kinect till TV-spelskonsolen Xbox) så fortsätter
gestigenkänning vara något som
kräver extern och ofta dyr hårdvara eller mycket kontrollerade miljöer för att
fungera bra. Detta problem kan endast lösas genom att implementera ett
videobaserat system som är tillräckligt robust för att kunna identifiera
handgester i mycket varierande miljöer.

Denna rapport behandlar både hudigenkänning och gestigenkänning, och ger en
lösning på gestigenkänningsproblemet i dess enklaste form, när
datorn endast förväntas känna igen statiska gester. Detta görs med hjälp av
Gaussian Mixture Models för hudklassificering och \knn-metoden (k
närmsta grannar) för tolkning
av handgester utifrån en mängd egenskaper som extraheras ur bilddata.
Vidare lägger den en
teoretisk grund för att även kunna känna igen rörliga gester med hjälp av
så kallade dolda Markovmodeller.


\end{document} 



\begin{thebibliography}{FFF}
\bibitem[UR]{rapp} Utformning av rapporter och kandidatarbetens skriftliga presentation för Civilingenjörsprogrammen vid Chalmers tekniska högskola. 2008. Göteborg: Chalmers Tekniska Högskola
\end{thebibliography}

\end{document}                 % The input file ends with this command.
