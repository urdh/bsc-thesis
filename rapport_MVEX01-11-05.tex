\documentclass[a4wide]{article}
\usepackage[utf8]{inputenc}

\usepackage[T1]{fontenc}
\usepackage{lmodern}

\usepackage[british,swedish]{babel}
\usepackage{graphicx}
\usepackage{ifpdf}
\ifpdf\DeclareGraphicsExtensions{.pdf,.png,.jpg}\fi
\usepackage{amsmath, amssymb}
\usepackage{epsfig}
%\usepackage{floatflt}
%\usepackage{ae}
\usepackage{icomma}
\usepackage{subfig}
\usepackage{subfiles}
\usepackage{fancyhdr}
\usepackage{hyperref}
\urlstyle{same}
\usepackage{listings}
\usepackage{textcomp}
\usepackage{harvard}

% Fina inställningar
\citationmode{abbr}
\numberwithin{equation}{subsection}

\begin{document}

% Titelsida
\subfile{topmatter.tex}
\newpage
% Inledning
\subfile{kapitel/inledning.tex}
\newpage
% Huvuddel
% Ändra inbördes ordning på allt det här så att det blir logiskt.
% Ta även bort kapitel som vi inte ska ha, och lägg till nya.

\section{Teori}
\subfile{kapitel/fargrymder.tex}
\subfile{kapitel/hudklassificering.tex}
%\subfile{kapitel/bildanalys.tex}

\newpage
% Slutsats/Diskussion

\newpage
% Källförteckning
\bibliographystyle{dcumod}
\bibliography{referenser}
\newpage
% Bilagor
\appendix
%\chapter{Bilaga}

\end{document}

% Chalmers referensguide rekommenderar Harvard-stil (författare/årtal):
%  http://www.lib.chalmers.se/education/guides/references/
%
% Följande kommandon finns tillgängliga för att referera:
% \citeasnoun{..}    ger   "Författare (2000)"
% \cite{..}          ger   "(Författare, 2000)"
% \possesivecite{..} ger   "Författares (2000)"
% \citeaffixed{..}{t.ex.} ger "(t.ex. Författare, 2000)"
%
% Man kan även referera till specifika sidor: \cite[s.~32]{..}
