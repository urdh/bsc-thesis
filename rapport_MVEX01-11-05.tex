\documentclass[a4wide,twoside,draft]{article}
\usepackage[utf8]{inputenc}

\usepackage[T1]{fontenc}
\usepackage{lmodern}

\usepackage[british,swedish]{babel}
\usepackage{graphicx}
\usepackage{ifpdf}
\ifpdf\DeclareGraphicsExtensions{.pdf,.png,.jpg}\fi
\usepackage{amsmath, amssymb}
\usepackage{epsfig}
%\usepackage{floatflt}
%\usepackage{ae}
\usepackage{icomma}
\usepackage{subfig}
\usepackage{subfiles}
\usepackage{fancyhdr}
\usepackage{hyperref}
\urlstyle{same}
\usepackage{listings}
\usepackage{textcomp}
\usepackage{harvard}
\usepackage[usenames,dvipsnames]{color}
\usepackage{fancyhdr}

% Titel här istället för topmatter.tex för att kunna använda i headers
\newcommand{\titel}{Statistisk bildanalys för människa-dator-interaktion} %Skriv in titeln här
\newcommand{\undertitel}{Analys av handgester med hjälp av dolda Markovmodeller} %Skriv in undertiteln här, eller lämna tomt

% Fina inställningar
\citationmode{abbr}
\definecolor{light-gray}{gray}{0.95}
\renewcommand{\thefootnote}{\fnsymbol{footnote}}
\newcommand{\notes}[1]{{\color{BrickRed}\textbf{Anteckning: }\color{Red}#1}}
%\renewcommand{\sectionmark}[1]{\markright{\thesection.\thesubsection~#1}{}}
\renewcommand{\subsectionmark}[1]{\markright{\thesubsection~\textit{#1}}{}}
\fancypagestyle{plain}{
  \fancyhf{} % remove everything
  \fancyhead[LE,RO]{\thepage}  
  \renewcommand{\headrulewidth}{0.4pt}
  \renewcommand{\footrulewidth}{0pt}
}
\fancypagestyle{fancier}{
  \fancyhf{}
  \fancyhead[LE,RO]{\thepage}
  \fancyhead[RE]{\textsc{\titel}}
  \fancyhead[LO]{\rightmark}
  \renewcommand{\headrulewidth}{0.4pt}
  \renewcommand{\footrulewidth}{0pt}
}
\lstset{
  language=Matlab,
  basicstyle=\normalsize
  numbers=left,
  numberstyle=\footnotesize,
  stepnumber=2,
  numbersep=5pt,
  backgroundcolor=\color{light-gray},
  tabsize=4,
  breaklines=true,
  breakatwhitespace=false,
  escapeinside={\%*}{*)},
  morekeywords={}
}

\begin{document}

% Titelsida
\subfile{topmatter.tex}
\newpage

% Förord
\pagenumbering{Roman}
\pagestyle{plain}
\section*{Förord}
\subfile{kapitel/forord.tex}
\newpage

\tableofcontents
\newpage
\pagestyle{fancier}

% Inledning
\pagenumbering{arabic}
\section{Inledning}
\subfile{kapitel/inledning.tex} %section
\newpage

% Huvuddel
% Ändra inbördes ordning på allt det här så att det blir logiskt.
% Ta även bort kapitel som vi inte ska ha, och lägg till nya.
\section{Teori}
\subfile{kapitel/fargrymder.tex}
\subfile{kapitel/hudklassificering.tex}
\subfile{kapitel/bildanalys.tex}
\subfile{kapitel/features.tex}
\newpage

% Slutsats/Diskussion
\section{Resultat}
%\subfile{kapitel/???.tex}
\newpage

% Källförteckning
\pagestyle{plain}
\bibliographystyle{dcumod}
\bibliography{referenser}
\newpage

% Bilagor
\appendix
%\chapter{Bilaga}

\end{document}

% Chalmers referensguide rekommenderar Harvard-stil (författare/årtal):
%  http://www.lib.chalmers.se/education/guides/references/

% Följande kommandon finns tillgängliga för att referera:
% \citeasnoun{..}    ger   "Författare (2000)"
% \cite{..}          ger   "(Författare, 2000)"
% \possesivecite{..} ger   "Författares (2000)"
% \citeaffixed{..}{t.ex.} ger "(t.ex. Författare, 2000)"
%
% Man kan även referera till specifika sidor: \cite[s.~32]{..}
