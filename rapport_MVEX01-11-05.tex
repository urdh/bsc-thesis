\documentclass[a4wide,twoside]{article}
\usepackage[utf8]{inputenc}

\usepackage[T1]{fontenc}
\usepackage{lmodern}

\usepackage[british,swedish]{babel}
\usepackage{graphicx}
\usepackage{ifpdf}
\ifpdf\DeclareGraphicsExtensions{.pdf,.png,.jpg}\fi
\usepackage{amsmath, amssymb}
\usepackage{epsfig}
%\usepackage{floatflt}
%\usepackage{ae}
\usepackage{icomma}
%\usepackage{subfig}
\usepackage{subfiles}
\usepackage{fancyhdr}
\usepackage{hyperref}
\urlstyle{same}
\usepackage{moreverb}
\usepackage{textcomp}
\usepackage{harvard}
\usepackage[usenames,dvipsnames]{color}
\usepackage{fancyhdr}
\usepackage{paralist}
\usepackage{parskip}
\usepackage{booktabs}
\usepackage{xspace}
\usepackage[font=small,format=plain,labelfont=bf,textfont=it]{caption,subfig}
%\usepackage{afterpage}
\usepackage{slashbox}

% Titel här istället för topmatter.tex för att kunna använda i headers
\newcommand{\titel}{Statistisk bildanalys av handgester för människa-dator-interaktion} %Skriv in titeln här
\newcommand{\undertitel}{} %Skriv in undertiteln här, eller lämna tomt

% Fina inställningar
\frenchspacing
\citationmode{abbr}
\setcounter{secnumdepth}{3} % Osäker på dessa två..
\setcounter{tocdepth}{3}    % Vad blir snyggast - med eller utan?
\definecolor{light-gray}{gray}{0.95}
\let\oldparagraph=\paragraph
\let\oldsubparagraph=\subparagraph
\renewcommand{\paragraph}[1]{\oldparagraph[#1]{#1.}}
\renewcommand{\subparagraph}[1]{\oldsubparagraph[#1]{\normalfont\itshape#1.}}
\renewcommand{\thefootnote}{\fnsymbol{footnote}}
\newcommand{\notes}[1]{{\color{BrickRed}\textbf{Anteckning: }\color{Red}#1}}
\renewcommand{\subsectionmark}[1]{\markright{\thesubsection~\textit{#1}}{}}
\fancypagestyle{plain}{
  \fancyhf{} % remove everything
  \fancyhead[LE,RO]{\thepage}  
  \renewcommand{\headrulewidth}{0.4pt}
  \renewcommand{\footrulewidth}{0pt}
}
\fancypagestyle{fancier}{
  \fancyhf{}
  \fancyhead[LE,RO]{\thepage}
  \fancyhead[RE]{\textsc{\titel}}
  \fancyhead[LO]{\rightmark}
  \renewcommand{\headrulewidth}{0.4pt}
  \renewcommand{\footrulewidth}{0pt}
}
\newenvironment{code}[0]{\medskip\center\boxedverbatim}{\endboxedverbatim\endcenter}
%\afterpage{\clearpage}
\expandafter\def\expandafter\quote\expandafter{\quote\small}

% Kommandon
\newcommand{\vect}[1]{\ensuremath{\mathbf{#1}}}  % Vektorer
\newcommand{\N}{\ensuremath{\mathcal{N}}}     % Normalfördelning
\newcommand{\Prob}{\ensuremath{\mathrm{P}}}   % Sannolikhet av
\newcommand{\R}{\ensuremath{\mathbb{R}}}      % Reella tal
\newcommand{\rd}{\ensuremath{\mathrm{d}}}     % 'rakt' d
\newcommand{\id}{\ensuremath{\,\rd}}          % Integral-d
\newcommand{\MATLAB}{MAT\-LAB\xspace}         % MATLAB
\newcommand{\knn}{$k$-NN\xspace}              % kNN

\begin{document}

% Titelsida
\subfile{topmatter.tex}

% Förord
\pagenumbering{Roman}
\pagestyle{plain}
\section*{Förord}
\subfile{kapitel/forord.tex}
\cleardoublepage
\tableofcontents
\cleardoublepage
\pagestyle{fancier}

% Inledning
\pagenumbering{arabic}
\section{Inledning}\label{sec:inledning}
\subfile{kapitel/inledning.tex}
\subfile{kapitel/problem.tex}
\subfile{kapitel/avgransning.tex}

\cleardoublepage

% Huvuddel
% Ändra inbördes ordning på allt det här så att det blir logiskt.
% Ta även bort kapitel som vi inte ska ha, och lägg till nya.
\section{Teori}
\subfile{kapitel/klassificering.tex}
\subfile{kapitel/hudklassificering.tex}
\subfile{kapitel/features.tex}
\subfile{kapitel/knn.tex}
\subfile{kapitel/HMM.tex}
\cleardoublepage

\section{Metod}
%\subfile{kapitel/metod.tex}
\subfile{kapitel/metod_gester.tex}
\subfile{kapitel/metod_hud.tex}
\subfile{kapitel/metod_datainsamling.tex}
%meffe: Generering av kodbok
%david: test av kNN
\cleardoublepage

\section{Resultat}
\subfile{kapitel/resultat.tex}
\subfile{kapitel/resultat_hudklassificering.tex}
\subfile{kapitel/resultat_features.tex}
\subfile{kapitel/resultat_knn.tex}
\cleardoublepage

\section{Diskussion}
%\subfile{kapitel/diskussion.tex}
%\subfile{kapitel/diskussion_xxx.tex}
%\subfile{kapitel/diskussion_yyy.tex}

\cleardoublepage

% Källförteckning
\pagestyle{plain}
\bibliographystyle{dcumod}
\bibliography{referenser}
\newpage

% Bilagor
\appendix
%\chapter{Bilaga}

\end{document}

% Chalmers referensguide rekommenderar Harvard-stil (författare/årtal):
%  http://www.lib.chalmers.se/education/guides/references/

% Följande kommandon finns tillgängliga för att referera:
% \citeasnoun{..}    ger   "Författare (2000)"
% \cite{..}          ger   "(Författare, 2000)"
% \possesivecite{..} ger   "Författares (2000)"
% \citeaffixed{..}{t.ex.} ger "(t.ex. Författare, 2000)"
%
% Man kan även referera till specifika sidor: \cite[s.~32]{..}
