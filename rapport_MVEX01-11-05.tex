\documentclass[a4wide]{article}
\usepackage[utf8]{inputenc}
%\usepackage[T1]{fontenc}
\usepackage[british,swedish]{babel}
\usepackage{graphicx}
\usepackage{ifpdf}
\ifpdf\DeclareGraphicsExtensions{.pdf,.png,.jpg}\fi
\usepackage{amsmath, amssymb}
\usepackage{epsfig}
%\usepackage{floatflt}
%\usepackage{ae}
\usepackage{icomma}
\usepackage{subfig}
\usepackage{subfiles}
\usepackage{fancyhdr}
\usepackage{hyperref}
\usepackage{listings}
\usepackage{textcomp}
%\usepackage{harvard}

\begin{document}

\subfile{topmatter.tex}
\subfile{kapitel/forord.tex}
\newpage
\subfile{kapitel/inledning.tex}

testar att referera till Davies: \cite{Davies05}
med \verb#\nocite{Shapiro00}# kan man lägga till i bibliografin utan att citera.

% Chalmers referensguide rekommenderar Harvard-stil (författare/årtal):
%  http://www.lib.chalmers.se/education/guides/references/
% 
% Följande kommandon finns tillgängliga för att referera:
% \citeasnoun{..}    ger   "Författare (2000)"
% \cite{..}          ger   "(Författare, 2000)"
% \possesivecite{..} ger   "Författares (2000)"
% \citeaffixed{..}{t.ex.} ger "(t.ex. Författare, 2000)"
%
% Man kan även referera till specifika sidor: \cite[s.~32]{..}

\bibliographystyle{alpha}
\bibliography{referenser} % importerar referenser.bib

\end{document}
