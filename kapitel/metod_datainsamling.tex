\documentclass[../rapport_MVEX01-11-05]{subfiles}
\begin{document}

\section{Datainsamling och analys}

\subsection{Inspelning av gester}
Eftersom syftet med arbetet var att behandla rörliga
gester var det naturligt att 
använda sig av inspelade filmsekvenser av varje gest.
För att bearbetningen av av dessa filmsekvenser inte skulle ta för lång tid,
och för att samma metod senare skulle kunna tillämpas i realtid,
nöjde vi oss med en upplösning på $320\times240$ bildpunkter.

Kameran var en Logitech C250, med automatisk kompensering av vitbalans,
vilker medförde att färger i kläder och omgivning kraftigt påverkade bilden,
och därmed förvrängde hudfärgerna. Detta gjorde i sin tur att vi var
tvungna att ha ett större område i hudfärgkartan
som motsvade hud, varpå fler bildområden felaktigt klassificerades som hud.
För att råda
bot på detta valde vi att begränsa uppställningen till en miljö fri från
hudliknande färger och utan varierande ljusförhållanden.

Vi lät åtta personer spela in varje gest fem gånger.
För att inom varje gest erhålla större spridning av handens position och form
visade försökspersonen de olika gesterna efter varandra, och upprepade sedan
hela sekvensen tills alla filmats fem gånger. Sammanlagt fick vi då 480
filmklipp att analysera.
Alla filmer verifierades för hand och uppenbart felaktiga ersattes, varefter
de namngavs med en gestspecifik bokstavskod och ett tal mellan 1 och 40.

\subsection{Egenskapsanalys av filmerna}
Egenskaperna från kapitel \ref{sec:features} samlades i en funktion
(bilaga~\ref{sec:matlab:features}) där MATLABs inbyggda funktioner kombinerades
med egenskrivna.
Dessa egenskaper beräknades
för den isolerade handen i alla bildrutor från varje film och resultatet
sparades i matriser.

I samband med detta normerades också dessa
matriser mot datamängden så att egenskaperna blev $\N(0,1)$-fördelade,
under antagandet att de var normalfördelade från början (vilket inte
nödvändigtvis är sant).

\end{document}
