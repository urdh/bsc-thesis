\documentclass[../rapport_MVEX01-11-05]{subfiles}
\begin{document}

\section{Datainsamling och analys}

\subsection{Inspelning av gester}
Eftersom det ursprungliga syftet med arbetet var att behandla även rörliga
gester var det naturligt att använda sig av korta filmsekvenser för varje gest.
För att bearbetningen inte skulle ta för lång tid nöjde vi oss med en
bildupplösning på $320\times240$ bildpunkter.

Kameran var en Logitech C250, med inbyggd kompensering av vitbalans. Detta
medförde att omgivningens färger, samt färger i kläder kraftigt störde bildens 
färg. Detta medförde att vi var tvungna att ha ett stort område i hudfärgkartan
som motsvade hud, och därmed en så ren omgivning som möjligt.

Vi lät åtta personer spela in varje gest fem gånger.
För att inom varje gest erhålla större spridning på handens position och form
visade försökspersonen de olika gesterna efter varandra, och upprepade sedan
hela sekvensen tills alla filmats fem gånger. Sammanlagt fick vi då 480
filmklipp att analysera.

Alla filmer verifierades för hand och uppenbart felaktiga ersattes. Därefter
namngavs de efter gest med en bokstavskod samt ett tal från 1 till 40.

\subsection{Egenskapsanalys av filmerna}
För att snabbare kunna arbeta med materialet analyserade vi
egenskaperna för den isolerade handen i alla bildrutor i varje film och sparade
resultatet i en $m\times n$-matris, där $n$ är antalet egenskaper och $m$ är
antalet bildrutor i filmen. Varje sådan matris sparades sedan i en ''cell array''
i MATLAB. Detta gjordes med skript som itererade över filnamnen
med en specificerad hudfärgskarta. I samband med detta normerades också dessa
matriser mot datamängden så att egenskaperna blev normalfördelade.

\end{document}
