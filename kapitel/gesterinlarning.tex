\documentclass[../rapport_MVEX01-11-05]{subfiles}
\begin{document}
\subsection{Gester och inlärning}
\subsubsection{Våra gester}
Handgester kan delas in i två kategorier; statiska och dynamiska. De
statiska gesterna är sådana där handen inte rör sig, utan en enda
bildruta är tillräckligt för att se vad det är för gest. Exempel på
sådana är freds- och segergesten med två fingrar i ett V och tumme
upp. Dynamiska gester å andra sidan är sådana som innefattar en
rörelse, exempelvis en vinkning. Dessa är betydligt svårare att
behandla ur ett igenkänningsperspektiv, då de kräver en följd av
bildrutor och bilderna därför inte kan behandlas separat.

Vid testerna av vårt program har vi använt tre statiska gester och två
dynamiska:
\begin{enumerate}
\item Spock. Handflatan mot kameran, fingrarna sträckta med pek- och
  långfinger ihop, ring- och lillfinger ihop och mellanrum mellan
  lång- och ringfinger. Även kallad Vulcan Salute, från TV-serien Star
  Trek, inspirerad av en judisk hälsning.
\item Seger. Handflatan mot kameran, pek- och långfinger sträckta i
  ett V, övriga fingrar knutna. Kan även ses som en fredssymbol,
  flitigt använd av allierade och motståndsrörelser i de ockuperade
  länderna under andra världskriget.
\item Tumme upp. Sträckt tumme uppåt, i övrigt knutna fingrar. En
  positiv gest i Europa, men i stora delar av världen en förolämpning.
\item Hajen. Handflatan neråt, fingrarna sträckta, pekfinger och tumme
  mot kameran. Tummen och de andra fingrarna rör sig motsatt upp och
  ned medan handen rör sig mot bildens kant, som en haj som ''äter sig
  framåt''.
\item Slap. Handen sträckt, börjar med handflatan mot kameran och
  fingrarna horisontellt. Handleden böjs sedan så att handen ''fälls
  över'' och slutar med handryggen mot kameran, fingrarna fortfarande
  horisontellt. 
\end{enumerate}
\end{document}
