\documentclass[../rapport_MVEX01-11-05]{subfiles}
\begin{document}
%\subsection{Problem}
%\notes{Ordentlig problemformulering med tillräcklig information för att förstå teorikapitlet.}

%\notes{Tog bort \texttt{\textbackslash section\{Problem\}}, kan inte få den att passa in
%ordentligt och tycker att inledningen förklarar problemet hyfsat.}

\notes{%\subsection{Frågeställningar}
Rapporten kommer besvara följande konkreta frågeställningar:
\begin{enumerate}
    \item Hur isolerar man på bästa sätt hud i en bild?
    \item Hur identifierar man en hand utifrån identifierad hud?
		\item Vilka karakteristiska egenskaper hos handens geometri och rörelse
					kan beräknas?
    \item Hur tränar man en klassificerare till att kunna skilja på
					handgester utifrån dessa egenskaper?
    \item Hur bra är klassificeraren givet en viss uppsättning egenskaper?
\end{enumerate}
}

\end{document} 

