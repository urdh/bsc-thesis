\documentclass[../rapport_MVEX01-11-05]{subfiles}
\begin{document}

\subsection{Klassificering av hudfärg och urskiljning av händer}

Metoden för identifiering av händer visade sig fungera mycket väl i
''kontrollerad miljö''. De krav som ställdes var, förrutom långärmad
tröja, att belysningen skulle vara god och helst komma från ljusrör
och att bilden inte fick innehålla för stora rödfärgade partier. Detta
då sannolikhetsfördelningen för hudfärg är för stor åt det röda
hållet. 

En visualisering av hudurskiljningen kan ses i figur
\ref{fig:hudklassificering}, där alla områden som klassats som hud
gjorts vita och övriga svarta. Handen är markerad med en ram i figur
\ref{fig:handklassificering}.

\begin{figure}
	  \centering
		\label{fig:hudklassificering}
		\caption{Figuren visar de områden i bilden som
                  klassats som hud.}
%    \includegraphics{bilder/hudklassificering}
\end{figure}

\begin{figure}
	  \centering
		\label{fig:handklassificering}
		\caption{Handen har ringats in av programmet, som det
                  ''tillräckligt'' stora objekt som befinner sig
                  längst till vänster i bilden.}
%    \includegraphics{bilder/handklassificering}
\end{figure}

\end{document}
