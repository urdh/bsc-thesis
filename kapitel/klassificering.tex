\documentclass[../rapport_MVEX01-11-05]{subfiles}
\begin{document}
\subsection{Statistisk klassificering}\label{sec:klassificering}

Klassificering är ett viktigt statistiskt verktyg, speciellt när det gäller
kvantitativ analys. Beroende på tillfälle finns det olika metoder för
klassificering; enkla sannolikhetsbaserade metoder som Gaussian Mixture
Models eller mer deterministiska metoder så som \knn-metoden.

Klassificering handlar egentligen om att göra en kvalificerad gissning
i någon fråga, t.ex.~om en bildpunkt innehåller hud, eller om ett område har
en viss form. I denna mening måste man alltid ha ''träningsdata'' att jämföra med
till sin
klassificerare, även om det inte alltid handlar om faktisk träning. Angreppssätt
så som HMM, skogar av slumpmässiga träd, genetiska metoder och liknande kräver
träning i form av att känd data matas in i klassificeraren varpå denne
konstruerar en struktur som används för framtida klassificering.
Enklare metoder så som \knn 
kräver endast känd data att jämföra med; denna data är då prototypdata.

\notes{Behövs det mer här?}

\end{document} 

