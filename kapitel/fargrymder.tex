\documentclass[../rapport_MVEX01-11-05]{subfiles}
\begin{document}

\subsection{Färgrymder}\label{sec:fargrymder}

Färg kan beskrivas matematiskt som vektorer i ett färgrum, vars
basvektorer är färgkomponenter. För att kunna representera alla färger
behöver man ett tredimensionellt färgrum, men i vissa situationer kan
det räcka --- kanske till och med vara fördelaktigt -- med ett
tvådimensionellt färgrum \cite{Kakumanu07}.

\marginpar{Hur mycket ska vi förklara RGB, HSI et al. om vi inte
använder dem egentligen?}

Vilken färgrymd som är bäst för just hudklassificering beror på
situation. RGB-liknande färgrymder har använts av bland annat
\citeasnoun{Lockton02} och \citeasnoun{Sebe04}, och ter sig lämplig
eftersom det är denna färgrymd som används i datorer. Perceptuella
färgrymder som separerar kulör och styrka, till exempel HSL, har inte
rönt någon större framgång eftersom de är prestandamässigt olämpliga
att använda i realtidstillämpningar.

Ortogonala färgrymder har däremot nått stor användning för
hudklassificering \cite{Hsu02,Elmezain08,Hassanpour08} eftersom de
både separerar färg från ljusstyrka och är linjärtransformationer av
RGB och därmed prestandamässigt lämpliga för realtidsapplikationer.
Det är en sådan färgrymd vi valt att använda: $YC_bC_r$.

\subsubsection[Färgrymden $\mathrm{YC_bC_r}$]{Färgrymden $\mathbf{YC_bC_r}$}

$YC_bC_r$-rymden är ett försök att ortogonalisera RGB-rymden genom en
linjärtransformation, vilken resulterar i en basvektor som
representerar ljusstyrka ($Y$) och två som representerar ''chroma''
(färg), $C_b$ och $C_r$. Enligt den standard International
Telecommunication Union publicerat \cite{ITU-BT601} ges
linjärtransformationen av

\begin{equation*}
  \label{eq:farg:ycbcr}
  \begin{gathered}
  Y   = 16  + ( 65.481R + 128.553G + 24.966B)\\
  C_b = 128 + (-37.797R - 74.203G  + 112.0B )\\
  C_r = 128 + (112.0R   - 93.786G  - 18.214B)
  \end{gathered}
\end{equation*}

där $R$, $G$, och $B$ är värden mellan $0$ och $255$ som representerar
färgens värde i RGB-rymden.

\end{document} 
