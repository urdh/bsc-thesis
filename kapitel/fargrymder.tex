\documentclass[../rapport_MVEX01-11-05]{subfiles}
\begin{document}

\subsection{Färgrymder}

Färg kan beskrivas matematiskt som vektorer i ett färgrum, vars
basvektorer är färgkomponenter. För att kunna representera alla färger
behöver man ett tredimensionellt färgrum, men i vissa situationer kan
det räcka med ett tvådimensionellt färgrum --- ibland kan det till och
med vara fördelaktigt att göra sig av med en dimension \cite{Kakumanu07}.

\subsubsection{Grundläggande färgrymder: RGB, rgb och rg}

I datorer används av tradition ofta RGB-färskalan, en \emph{additiv}
färgrymd med komponenterna röd, grön och blå. Av tekniska skäl brukar
man begränsa dessa komponenter till heltal i intervallet $(0,255)$
vilket gör att man kan representera totalt $16\:581\:375$ färger.
RGB-skalan är mycket praktisk när man ska visa bilderna på LCD- eller
CRT-skärmar, eftersom dessa opererar genom s.k. subpixlar med dessa
tre färger. \citeasnoun{Lockton02} använde RGB-rymden för huddetektion
med viss framgång.

Modellen beskriver dock inte färger på ett intuitivt sätt, och är inte
ortogonal, vilket ställer till med problem i många tillämpningar.

Det finns andra färgrymder som baseras på RGB-rymden; till exempel
\emph{rgb}-rymden (notera gemenerna), som är normaliserad RGB enligt
\eqref{eq:farg:rgb} och som använts i syfte att klassificera hud av bland
annat \citeasnoun{Sebe04}.

\begin{equation}
  \label{eq:farg:rgb}
  \begin{gathered}
  r = \frac{R}{R+G+B} \\
  g = \frac{G}{R+G+B} \\
  b = \frac{B}{R+G+B}
  \end{gathered}
\end{equation}

Den reducerade färgrymden \emph{rg}, egentligen en kromaticitetsrymd
(dvs. färgrymd utan intensitetsinformation) utgår från rgb-rymden men
kasserar den blå komponenten.

Transformationer av RGB-rymden till endimensionella rum genom enkla
aritmeriska operationer har även använts i syfte att klassificera hud;
\citeasnoun{Brand00} nådde viss framgång med ration $R/G$ och summan
$R/G+R/B+G/B$, men visade att en SPM (\emph{Skin Probability Map}) i
RGB-rymden var mer effektiv.

\subsubsection{Perceptuella färgrymder: HSI, HSV och HSL}

Ett mer naturligt sätt att beskriva färger är genom de tre
komponenterna \emph{hue} (kulör), \emph{saturation} (mättnad) och en
tredje komponent som beskriver ljusstyrkan (\emph{intensity},
\emph{value} eller \emph{lightness}). Även om detta är ett mer
intuitivt sätt att se färger på, så är det inte särskilt väl lämpat
för hudklassificering då det är relativt beräkningsintensivt att
konvertera till och från dessa färgrymder --- till skillnad från
övriga färgrymder som är linjärtransformationer av RGB-rymden, så
involverar konvertering till HSL flera beräkningssteg och
trigonometriska operationer.

\subsubsection[Ortogonala färgrymder: YUV och $\mathrm{YC_bC_r}$]{Ortogonala färgrymder: YUV och $\mathbf{YC_bC_r}$}

Ortogonala färgrymder grundar sig i linjärtransformationer av
RGB-rymden för att ortogonalisera rummet, och på så sätt kunna
representera informationen i tre variabler som är så oberoende som
möjligt. Både YUV och $YC_bC_r$ har en komponent Y som representerar
\emph{luma}, ljusstyrka, och två komponenter som representerar
\emph{chroma}, färg. Dessa färgrymder har rönt stor framgång inom
huddetektion \cite{Hsu02,Elmezain08,Hassanpour08}.

Vi har valt att använda $\mathrm{YC_bC_r}$, en enkel linjär transformation av
RGB-rymden enligt \eqref{eq:farg:ycbcr}. Denna är fördelaktig då den
är ortogonal, skiljer färg från ljusstyrka (något
som är mycket viktigt då man i stort endast är intresserad av färgen
när man ska hitta hudpixlar), och eftersom den är en \emph{enkel}
linjärtransformation, vilket gör det enkelt att implementera en
effektiv transformation från RGB --- en mycket viktig egenskap eftersom vi
behöver göra detta i realtid.

\begin{equation}
  \label{eq:farg:ycbcr}
  \begin{gathered}
  Y   = 16  + ( 65.481R + 128.553G + 24.966B)\\
  C_b = 128 + (-37.797R - 74.203G  + 112.0B )\\
  C_r = 128 + (112.0R   - 93.786G  - 18.214B)
  \end{gathered}
\end{equation}

Det ska även nämnas att det finns många olika färgrymder som kallas
$YC_bC_r$; dessa skiljer sig på konstanter i linjärtransformationen.
Den standard som beskrivs av \eqref{eq:farg:ycbcr} är \emph{ITU-R BT.601}.

\end{document} 
