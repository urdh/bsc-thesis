\documentclass[../rapport_MVEX01-11-05]{subfiles}
\begin{document}

\subsection{Bakgrund}

I dagens informationssamhälle blir det svårare och svårare
att undvika kontakt med datorer. Vi omges av allt från uppenbara fall såsom
mobiltelefoner och digitalkameror till dolda styrsystem i bilar och den
automatiserade kaffekokaren på arbetsplatsen. Överallt finns det apparater
innehållande processorer som, både med och utan vår omedelbara vetskap,
underlättar vardagen vi lever i. Trots dessa framsteg kan man tycka att den
gängse uppfattningen av en dator är mycket konservativ, nämligen att den är en
låda kopplad till skärm, tangentbord och mus.

Med detta arbete ämnar vi undersöka samt implementera metoder som möjliggör
människa-dator-kommunikation via handgester, att skapa grunderna till ett
gränssnitt som kan uppfattas både intuitivt och i samklang med den snabba
utvecklingen omkring oss. Även om kommunikation via handgester utgör en
minoritet av kommunikationen i dagsläget, hävdar vi att ämnet är mer relevant än
någonsin, bl.a tack vare den senaste tidens explosion av spelsystem som nyttjar
tekniken. Ett framstående exempel är Microsofts Kinect med flera miljoner sålda
enheter \cite{kinect1}, vilket kan tänkas påvisa både intresse och behov av ett
mer intuitivt sätt att kommunicera med datorer.

Modern och innovativ kommunikation med datorer har länge varit en
synlig framtidsvision inom film- och TV-industrin, med relativt gamla
exempel som Hackers (1995) och Tron (1982) men även i mer moderna
företeelser såsom t.ex. CSI-serierna (2000--). Gestigenkänning --- kanske
framförallt den bildanalys som ingår i problemet --- är en nyckel i
många innovativa gränssnitt för kommunikation med datorer.

Bildbehandling och mönsterigenkänning kan även vara av stor nytta i både
övervakningsindustrin (för att känna igen mönster så som snatteri,
gatuvåld eller olaga intrång) och 
i bilindustrin, där man kan använda teknologin för att för att ta
kontroll över bilen vid eventuella farliga situationer, till exempel
genom att ''se'' när förare somnar.

Trots denna framgång kan det tyckas att visuell datorkommunikation inte har
haft det genomslag som förväntas av ett fält som funnits i flera decennier.
Detta faktum kan spegla de inneboende svårigheter som mönsterigenkänning bär
med sig. Att på ett tillförlitligt och effektivt sätt identifiera handgester,
något vi människor gör närmast omedvetet, förefaller vara mer invecklat än vad
man tror. En svårighet är bland annat den detaljrikedom människans kroppsspråk
bjuder på, subtila skillnader som inte kan klassificeras med lätthet. Dessa
problem kan även förvärras då programmet får missvisande indata. Ett pålitligt
program måste därför kunna prestera oberoende av miljöpåverkan, såsom ljussättning
och klädsel.

I detta arbete möter vi dessa utmaningar med metoder baserade på så kallad
diskriminantanalys. Analysen grundar sig i sannolikhetslära och matematisk
statistik och används för att finna unika drag, så kallade \emph{egenskaper} hos
specifika handgester, för att med hjälp av dessa skilja dem åt. De relevanta
matematiska problem som uppstår kräver, utöver kunskap inom dessa ämnen, även
bekantskap med optimeringslära.

\subsection{Problem och uppgift}

Våra problem kan grovt delas in i två kategorier: konstruktion och test.
Konstruktionsdelen handlar om att lyckas skapa ett program som ur en
filmsekvens, med hjälp av statistiska metoder, kan identifiera och särskilja ett
antal olika handgester. Testdelens fokus är att hitta en metod för att på ett
relevant sätt testa programmets träffsäkerhet och snabbhet.

\subsubsection{Konstruktion}

För att kunna identifiera handgesterna krävs
antagligen att ett antal olika \emph{punkter i bilden identifieras}, såsom
fingertoppar och punkter i handflatan. Detta är antagligen ett lurigt problem,
men bra algoritmer kan man hoppas att hitta i litteraturen.

Därefter måste vi hitta ett antal \emph{karakteristiska egenskaper} för de
gester vi valt, exempelvis olika avstånd och hastigheter.
Dessa används sedan till att \emph{klassificera gesterna} och identifiera vilken
av de olika gesterna som en viss videosekvens innehåller.

När detta är gjort ska vi hitta en metod att \emph{kontinuerligt söka gester} så
att vi kan identifiera gesterna i längre ''ointressanta''
videosekvenser, där även andra rörelser kan förekomma.

Man måste dessutom ta hänsyn till de subtila skillnader som finns
mellan olika personers sätt att visa gester, men kanske framförallt de
skillnader som finns mellan gester som utförs av \emph{en} person.
Detta är likt problem man kan stöta på i t.ex. handstilsigenkänning.
Till sist så vill vi att programmet ska kunna fungera
\emph{bra under svåra omständigheter} såsom i mörker, utan långärmade kläder och
oavsett hudfärg.

När detta är uppnått är nästa mål, att
kunna identifiera gesterna i \emph{realtid}. Först när detta är gjort blir
programmet användbart.
Om vi ej får tillräckligt bra prestanda med MATLAB kan vi inte utföra de
beräkningar som krävs för att behandla en videoström i realtid. Programmet
kommer då endast att kunna arbeta på inspelade filer, och om vi vill ha
visuell återkoppling från programmet så måste vi spara nya videofiler som
inspekteras i efterhand.

\subsubsection{Test och träning}

Funktionaliteten i programmet kommer att bygga på en stor datamängd extraherad
ur ''typgester''. Att kunna göra denna träning så tidseffektivt som möjligt är 
därför viktigt.

Det är även viktigt att vi testar programmet med flera olika personer,
och på så sätt bygger in ett stöd för skillnader mellan individuella
personer redan i träningsdatan.

Programmet ska sedan jämföras med andra program, eller kodbibliotek. Detta för
att se hur väl vår kod det fungerar i förhållande till befintlig programvara,
både vad gäller snabbhet och träffsäkerhet.


\end{document}