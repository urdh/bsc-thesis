\section{Hud- och handidentifiering}

\subsection{Klassificering av hud}
I 44 bilder tagna med webkameran separerades huden från resten av
bilden manuellt med hjälp av fotoredigeringsprogramvara. Baserat på
dessa pixlar bestämdes sedan en sannolikhetsfördelning för både hud
och icke-hud. Dessa fördelningar jämfördes sedan med de presenterade
av \citeasnoun{Elmezain08}.

Kvoten $\Prob(\textrm{hud})/\Prob(\textrm{ickehud})$ (se
\ref{sec:hudklassificering}), som sattes
manuellt, anger hur mycket vanligare bildpunkter med ickehud är i
bilderna. Även denna korrigerades för att nå en tillräcklig träffandel.

Sannolikheten för att en viss färg uppkommit från hud respektive
icke-hud är dock tidskrävande att räkna ut, varför beräkningen
är direkt olämplig att göra för varje bildpunkt i varje bild i en
filmsekvens. Detta lötses genom att göra beräkningen en enda gång för
varje tänkbar färg ($256^2$ stycken i $C_bC_r$-rymden). Resultatet sparades
sedan i en $256\times256$-matris, där färgvärdet används som koordinater.
I denna matris, eller ''hudfärgskarta'', kan man sedan slå upp
färgvärden för att direkt få veta sannolikheten för att dessa är
hud. Metoden har visats vara effektiv av bl.a.~\citeasnoun{Brand00}

\subsection{Urskiljning av handen}\label{sec:metod_hud:urskiljning}

När väl hudbildpunkterna i figuren urskiljts återstår att plocka ut
vad som är handen som ska följas. Det är troligt att man bland
bildpunkterna fått med ett antal mindre områden som felaktigt bedömts vara
hud. Dessutom är huvudet ofta med i bilden, vilket kan ställa till med
problem. 

Vår lösning på problemet är att först använda den inbyggda
matlabfunktionen \texttt{bwareaopen}, som är en inbyggd morfologisk
öppningsfunktion. Metoden beskrivs i
\ref{sec:morph} använder ett strukturelement $S$, som \texttt{bwareaopen}
genererar utifrån en parameter som specificerar
dess area --- vad som är en lämplig area
varierar med upplösning och
och avstånd till kameran. Vi använder en gräns på 200
bildpunkter. Detta fungerar bra även för relativt stort avstånd till
 en kamera med upplösning på $320\times240$ bildpunkter. Därefter loopar vi
igenom bildens kolonner från
höger för att se om kolonnen innehåller någon hudbildpunkt. Den första
bildpunkt som nås på detta sätt väljs till att tillhöra handen, och det
sammanhängande område som bildpunkten är en del av betraktas som den sökta
handen. Att på detta sätt välja objektet längst till höger av de
intressanta har tidigare gjorts av bl.a. \citeasnoun{Nielsen04}. De använde
sig dock av det högra av de två största sammanhängande objekten,
vilket har fördelen att man inte behöver sätta en manuell gräns för
när ett objekt är ''tillräckligt stort''. En nackdel är att man tvingas
söka igenom hela bilden, att den är något svårare att
implementera samt att den endast fungerar då både huvud och hand finns
i bilden.

En alternativ metod för att leta från
vänster är lätt att implementera om man vill kunna använda programmet
som vänsterhänt --- detta kommer dock att kräva en
separat inlärningsmängd till gestigenkänningen.

Metoden plockar med all hud som hänger samman med
handen, inklusive armen om personen framför kameran bär kortärmad tröja. För att
råda bot på detta kan man identifiera handleden och kapa handen
där. Detta har gjordes av \citeasnoun{Deimel99} genom att identifiera den största cirkel
som får plats helt i handen, som antas vara centrerad mitt i handen. Då
cirkeln förstoras något är det längsta cirkelsegment som helt får
plats i hudområdet placerat vid handleden, och mittpunktstangenten
till segmentet är då handleden. Handen utgörs då av hudområdet på den
sida om handleden som cirkeln befinner sig på. Denna metod har vi dock
inte implementerat, varför vi kräver att den filmade bär
långärmad tröja. 
