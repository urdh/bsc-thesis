\documentclass[../rapport_MVEX01-11-05]{subfiles}
\begin{document}
\begin{abstract}
    Syftet med arbetet är att skriva ett program som med hjälp av en
    enkel webbkamera kan analysera och identifiera ett antal handgester, och
    utföra kommandon därefter. Programmet skall vara kapabelt till att i realtid
    lokalisera händer, inklusive intressanta punkter och tillhörande mått. Dessa
    mått är de egenskaper som därefter skall bearbetas med metoder grundade
    i statistik och sannolikhetslära, för att med precision fastställa den aktuella
    gesten.

    Det slutgiltiga målet är ett program som är både stabilt och effektivt, med
    prestanda i klass med existerande program. Ett delmål är därför att ta fram ett
    prestandamått där vi på ett objektivt sätt kan jämföra vårt resultat med
    befintliga metoder.
\end{abstract}
\end{document}
