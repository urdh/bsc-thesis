\documentclass[../rapport_MVEX01-11-05]{subfiles}
\begin{document}

\subsection{Dolda Markovmodeller (HMM)}

En Hidden Markov model (HMM) är en stockastisk signalmodell som
beskriver ett system bestående av ett ändligt antal tillstånd. Likt en
vanlig Markovmodell förflyttar man sig sedan efter en given
sannolikhetsfördelning mellan tillstånden. Det som gör HMMs "hidden",
eller "dolda", är det faktum att varje tillstånd i sin tur genererar
symboler efter ännu en sannolikhetsfördelning. Då man i standardfallet
observerar modellens tillstånd direkt, observaras nu istället de
symboler som genereras då tillstånden växlar, symboler som motsvarar
den verkliga process man önskar modellera. Ett förtydligande exempel
lyder som följer: 

En person har ett antal tärningar, vissa är viktade och avviker därmed
från den annars uniforma fördelningen. Han kastar en tärning ur sin
samling och berättar för en kompis hur många ögon tärningen
visar. Därefter avgör tärningskastaren om han vill byta tärning eller
inte, och upprepar proceduren. Kompisen får alltså inte se vilken av
tärningarna som kastas för tillfället, utan får bara ta del av vad
respektive tärning visar. Denna process skulle kunna beskrivas med
hjälp av en HMM där varje enskild tärning representerar ett
tillstånd. Den förstnämnda personen rör sig mellan tärningarna
(tillstånden) efter eget behag och tillkännager enbart värdet
(symbolerna) som genereras av dem. 

Problematiken består nu i att konstruera den HMM som bäst beskriver
processen.  

HMM tillämpat på röst- och gestigenkänning är inget nytt. Redan i
slutet av 60-talet presenterade Baum och hans kollegor den
grundläggande teorin (Baum66-72), och i början av 70-talet
implementerades modellen i samband med röstigenkänning av både Baker
(\cite{Baker75}) och av Jelinek med kollegor (Jelinek69). I slutet av
80-talet presenterades en sammanfattande artikel om röstigenkänning
med hjälp av HMM av Rabiner (Rabiner89), som till stor del ligger till
grund för följande teoretiska framställning. Tillämpningar gentemot
gestigenkänning gjordes bland annat av Yamato och kollegor i början av
90-talet (Yamato92), och av Elmezain och kollegor så sent som 2008
(Elmezain08). (Kan även nämna Starner98, Rigol98)

\subsubsection{Beståndsdelarna av en HMM}

En HMM karakteriseras av följande:
\begin{enumerate}
\item N, antalet tillstånd i modellen. Vi betecknar de olika
  tillstånden som $S = \{S_1, S_2, \dots, S_N\}$, och tillståndet vid
  tiden $t$ som $q_t$.
\item M, antalet distinkta observationssymboler för ett givet
  tillstånd. De individuella symbolerna beteckner vi som $V =
  \{v_1,v_2,\dots,v_M\}$.
\item A, en övergångsmatris där varje element ger sannolikheten att
  vandra från ett tillstånd till ett annat, 
\begin{equation*}
a_{ij} = \textbf{P}[q_{t+1} = S_j | q_t = S_i],\quad 1 \leq i,j \leq N.
\end{equation*}
I fallet då man kan nå alla tillstånd från vilket tillstånd som helst
med ett enda steg har vi alltså $a_{ij} > 0$ för alla $i,j$. 
\item Sannolikhetsfördelning för observationssymboler givet ett
  specifikt tillstånd $j$, $B = b_j(k)$, där 
\begin{align*}
b_j(k) = \textbf{P}[v_k \text{ vid tiden } t|q_t = S_j] \quad &1 \leq j \leq N\\
&1 \leq k \leq M.
\end{align*}
$B(j,k)$ ger med andra ord sannolikheten att symbolen $v_k$ genereras
då vi befinner oss i tillstånd $j$.
\item Sannolikhetsfördelning för begynnelsetillstånd, $\pi =
  \{\pi_i\}$, där
\begin{equation*}
\pi_i = \textbf{P}[q_1 = S_i].
\end{equation*}
\end{enumerate}
Givet parametrarna $M$ och $N$, en specificering av symboler, samt de
tre sannolikhetsmåtten $A, B$ och $\pi$ är det alltså möjligt att
fullständigt specificera en HMM. I fortsättningen kommer vi att
använda oss av den något mer kompakta notationen $\lambda = (A,B,\pi)$
för att representera parametrarna för en given HMM.

\subsubsection{Grundläggande problem}

För att kunna föra över modellen från teori till praktisk tillämpning
återstår det dock en del. Huvudsakligen gäller det att finna lösningen
följande två problem. (Rabiner) (Jack Ferguson) (Rabiner presenterar
tre grundläggande frågor för implementerandet av en HMM, varav två är
av intresse för oss.)
\begin{description}
\item[Problem 1:] Givet en sekvens av observationer
  $\{O_1,O_2,\dots,O_t\}$ och en modell $\lambda = (A,B,\pi)$, hur
  beräknar man $\textbf{P}(\textbf{O}|\lambda)$, dvs sannolikheten att
  sekvensen genererades av modellen $\lambda$? Mer därtill, hur kan
  det göras så effektivt som möjligt?, en fråga som skall visa sig
  vara ytterst viktig.  
\item[Problem 2:] Hur går man tillväga för att optimera
  $\textbf{P}(\textbf{O}|\lambda)$ genom att justera
  modellparametrarna $\lambda = (A,B,\lambda)$? 
\end{description}

Kan man lösa det första problemet har man möjlighet att uppskatta hur
väl en modell matchar en viss observationsföljd. Man kan se det som
ett sätt att poängsätta, och på så sätt jämföra, olika modeller och
deras relation till en specifik observationsföljd. Denna synvinkel är
extremt användbar då man ställs inför utmaningen att välja mellan
flera existerande modeller. 
Det andra problemet behandlar modelloptimering; hur kan man med en
grundmodell och en observationsföljd optimera modellen på så sätt att
sannolikheten att den givna observationsföljden genereras blir
maximal? Lösningen till detta problem utgör en fundamental del av
vägen till att på ett framgångsrikt sätt implementera HMMs i
praktiken. Med hjälp av en "träningssekvens", alltså en följd av
observationer, kan man sedan "träna" en HMM. Vikten av att
optimeringsprocessen ger bra resultat är stor eftersom
träningssekvensen är den främsta kopplingen till det verkliga fenomen
man önskar modellera.      

\subsubsection{Framåt-bakåt-tekniken}

(Använda bilder i denna sektion?)

En användbar metod introducerades av Baum och hans kollegor, kallad
\emph{the Forward-Backward procedure} (referera till Baum67): vi
definierar framåtvariabeln $\alpha_t(i)$ som 

\begin{equation*}
\alpha_t(i) = \textbf{P}(O_1,O_2,\dots,O_t,q_t = S_i | \lambda),
\end{equation*}

alltså sannolikheten att ha genererat observationsföljden $\textbf{O}
= \{O_1,O_2,\dots,O_t$\}, samt befinna sig i tillstånd $S_i$ vid tiden
$t$, givet en modell $lambda$. En av metodens fördelar är att man med
hjälp av induktion kan finna $\alpha_t(i)$. Processen utgörs av
följande tre steg: 

\begin{enumerate}
\item Initialisering:

\begin{equation*}
\alpha_1(i) = \pi_iB_i(O_1), \quad 1\leq i \leq N
\end{equation*}

\item Induktion:

\begin{align*}
\alpha_{t+1}(j) =
\left[~\sum_{i=1}^N\alpha_t(i)A_{ij}~\right]B_j(O_{t+1}) \quad &1 \leq t \leq T-1 \\
&1 \leq j \leq N
\end{align*}

\item Avslutning:

\begin{equation*}
\textbf{P}(\textbf{O}|\lambda) = \sum_{i=1}^N\alpha_T(i)
\end{equation*}
\end{enumerate}

I steg 1), då $t=1$, initialiseras framåtvariabeln $\alpha_1(i)$ som
sannolikheten att börja i tillstånd $S_i$ och därefter generera
observation $O_1$. För $t=2$ använder man induktionssteget. Utgår man
från steg 1) kan man övertyga sig om att
$\sum_{i=1}^N\alpha_1(i)A_{ij}$ måste vara den totala sannolikheten
att hamna i $S_j$ vid $t=2$, efter att ha genererat $O_1$ i föregående
tillstånd. Efter multiplikation med $B_j(O_{2})$ fås därefter
sannolikheten att generera $O_2$, då man befinner sig i $S_j$ med en
observerad symbol $O_1$, eller med andra ord, $\textbf{P}(O_1,O_2,S_j
= q_2 | \lambda) = \alpha_2(j)$. Fortsätter man på samma vis har man
slutligen $\alpha_T(i) = \textbf{P}(O_1,O_2,\dots,O_T,S_i = q_T |
\lambda)$. Steg 3) ger slutligen svaret på problem 1, nämligen
sannolikheten $\textbf{P}(\textbf{O}|\lambda)$. Detta inses lätt då vi
använder oss av definitionen av $\alpha_T(i)$, 

\begin{equation*}
\textbf{P}(\textbf{O}|\lambda) = \sum_{i=1}^N\alpha_T(i) =
\sum_{i=1}^N\textbf{P}(O_1,O_2,\dots,O_T,q_T = S_i|\lambda). 
\end{equation*} 

Med andra ord är sannolikheten att generera observationsföljden
$\textbf{O}$ summan av sannolikheterna att generera $\textbf{O}$ med
sluttillstånd $S_i$, $1 \leq i \leq N$.

\subsubsection{Iterativ metod för träning av HMM}

Med en lösning till problem 1 i bagaget vänder vi oss till problem 2,
själva träningen av en HMM. Detta är ett betydligt mer invecklat
problem än det tidigare, med betydligt fler angreppsvinklar. En
analytisk lösning till optimeringsproblemet är inte känd, utan man
tvingas istället använda sig av Baum-Welch-metoden eller
gradient-tekniker. (Dempster77)(Levinson73). Det är rentav så att
givet en ändlig observationssekvens så finns det inget optimalt sätt
att optimera modellens parametrar. Däremot kan man välja $\lambda =
(A,B,\pi)$ på så sätt att $\textbf{P}(\textbf{O}|\lambda)$ är lokalt
maximerad. 


(BAUM WELCH) På liknande sätt som för framåtvariabeln $\alpha_t(i)$
definierar vi nu bakåtvariabeln $\beta_t(i)$: 

\begin{equation*}
\beta_t(i) = \textbf{P}(O_{t+1},O_{t+2},\dots,O_T | q_t = S_i, \lambda).
\end{equation*} 

$\beta_t(i)$ är alltså sannolikheten för en delsekvens av
observationer då man startar i tillstånd $S_i$. Processen består i det
fallet av två steg:

\begin{enumerate}
\item Initialisering: 
\begin{equation*}
\beta_T(i) = 1, \quad 1 \leq i \leq N.
\end{equation*}
\item Induktion: 
\begin{align*}
\beta_t(i) = \sum\limits_{j=1}^NA_{ij}B_j(O_{t+1})\beta_{t+1}(j), \quad &t =
T-1,T-2,\dots,1 \\
&1 \leq i \leq N.
\end{align*}
\end{enumerate}  

Inledningsvis definieras $\beta_T(i)$ godtyckligt till $1$, och sedan
fungerar induktionssteget efter samma princip som framåtvariabeln
$\alpha$. Dessa variabler kan nu avslutningsvis användas tillsammans
för att bilda två slutgiltiga variabler, varav den första definieras som 

\begin{equation*}
\xi_t(i,j) = P(q_t = S_i, q_{t+1} = S_j|\textbf{O},\lambda).
\end{equation*}

Kort och gott sannolikheten att befinna sig i tillstånd $S_i$ och
därefter $S_j$. Tack vare framåt- och bakåtvariabeln $\alpha$
respektive $\beta$ kan man nu uttrycka $\xi_t(i,j)$ som

\begin{equation*}
\xi_t(i,j) = \frac{\alpha_t(i)A_{ij}b_j(O_{t+1})\beta_{t+1}(j)}{\textbf{P}(\textbf{O}|\lambda)}.
\end{equation*} 

Nämnaren tillkommer för att ge uttrycket ett korrekt sannolikhetsmått, eftersom 

\begin{equation*}
\sum_{i=1}^N\sum_{j=1}^N\alpha_t(i)A_{ij}b_j(O_{t+1})\beta_{t+1}(j) = \textbf{P}(\textbf{O}|\lambda).
\end{equation*}

Givet $\xi_t(i,j)$ är steget inte långt till den fjärde variabeln
$\gamma_t(i)$, uttryckt på följande vis:

\begin{equation*}
\gamma_t(i) = \sum_{j=1}^N\xi_t(i,j).
\end{equation*}

Summerar vi sedan $\gamma_t(i)$ över tidsindexet $t$ får vi en storhet
som kan tolkas som antalet gånger tillståndet $S_i$ har besökts,
(eller likvärdigt: antalet förflyttningar från tillståndet $S_j$), med
andra ord:

\begin{equation*}
\sum_{t=1}^{T-1}\gamma_t(i) = \text{antalet förväntade förflyttningar
  från tillstånd $S_i$.}
\end{equation*} 

På samma sätt kan summation över tidsindexet hos variabeln
$\xi_t(i,j)$ tolkas som 

\begin{equation*}
\sum_{t=1}^{T-1}\xi_t(i,j) = \text{antalet förväntade förflyttningar
  från tillstånd $S_i$ till $S_j$.}
\end{equation*}

(Då $\xi_t(i,j)$ inte är definierad för $t=T$ går summationen endast
till $t = T -1$.)

Den iterativa metoden som återuppskattar modellens parametrar $A,B$
och $\lambda$ kan nu skrivas som

\begin{equation*}
\bar{\lambda} = \text{förväntat antal gånger i tillstånd $S_i$ vid
  $t=1$} = \gamma_i(i).
\end{equation*}


\begin{multline*}
\bar{A}_{ij} = \\ = \frac{\text{förväntat antal förflyttningar från
    tillstånd $S_i$ till $S_j$}}{\text{förväntat antal förflyttningar
    från tillstånd $S_i$}} = \\ =
\frac{\sum_{t=1}^{T-1}\xi_t(i,j)}{\sum_{t=1}^{T-1}\gamma_t(i)}.
\end{multline*}

\begin{multline*}
\bar{B}_j(k) = \\ = \frac{\text{förväntat antal gånger i tillstånd $S_j$
    och observerandes symbol $v_k$}}{\text{förväntat antal gånger i
    tillstånd $S_j$}} = \\ = \frac{\sum_{\substack{t=1\\s.t~ O_t =
      v_k}}^{T-1}\gamma_t{j}}{\sum_{t=1}^{T-1}\gamma_t(j)}.
\end{multline*}

Vänsterleden fås alltså genom att ta den nuvarande modellen $\lambda =
(A,B,\pi)$, beräkna högerleden, och därigenom erhålla $\bar{\lambda} =
(\bar{A},\bar{B}, \bar{\pi})$. Baum och hans kollegor visade
(Baum68,Baker75) att denna iterativa process faktiskt resulterar
antingen i 1) ett lokalt maximum för $\textbf{P}(\textbf{O}|\lambda)$
(alltså att $\lambda = \bar{\lambda}$), eller 2) i en förbättrad
modell $\bar{\lambda}$ på så sätt att
$\textbf{P}(\textbf{O}|\bar{\lambda}) > \textbf{P}(\textbf{O}|\lambda)$. 

\subsubsection{Olika typer av HMMs}

Då man talar om olika typer av HMMs syftar man först och främst på
vilken typ av övergångsmatris som används. Hittills har vi bara nämnt
fallet där $A_{ij} > 0$ för alla $i,j$, även kallad en ergodisk
modell. (Närmare bestämt innebär ergodicitet i sammanhanget att varje
tillstånd kan nås från alla andra tillstånd inom ändlig tid.) Om vi
tänker oss en modell med 4 st tillstånd får vi följaktligen övergångsmatrisen

\begin{equation*}
\bar{B}_j(k) = \frac{\text{förväntat antal gånger i tillstånd $S_j$
    och observerandes symbol $v_k$}}{\text{förväntat antal gånger i
    tillstånd $S_j$}} = \frac{\sum_{\substack{t=1\\s.t~ O_t =
      v_k}}^{T-1}\gamma_t{j}}{\sum_{t=1}^{T-1}\gamma_t(j)}.
\end{equation*}

Vänsterleden fås alltså genom att ta den nuvarande modellen $\lambda =
(A,B,\pi)$, beräkna högerleden, och därigenom erhålla $\bar{\lambda} =
(\bar{A},\bar{B}, \bar{\pi})$. Baum och hans kollegor visade
(Baum68,Baker75) att denna iterativa process faktiskt resulterar
antingen i 1) ett lokalt maximum för $\textbf{P}(\textbf{O}|\lambda)$
(alltså att $\lambda = \bar{\lambda}$), eller 2) i en förbättrad
modell $\bar{\lambda}$ på så sätt att
$\textbf{P}(\textbf{O}|\bar{\lambda}) > \textbf{P}(\textbf{O}|\lambda)$. 

\subsubsection{Olika typer av HMMs}

Då man talar om olika typer av HMMs syftar man först och främst på
vilken typ av övergångsmatris som används. Hittills har vi bara nämnt
fallet där $A_{ij} > 0$ för alla $i,j$, även kallad en ergodisk
modell. (Närmare bestämt innebär ergodicitet i sammanhanget att varje
tillstånd kan nås från alla andra tillstånd inom ändlig tid.) Om vi
tänker oss en modell med 4 st tillstånd får vi följaktligen övergångsmatrisen

\begin{equation*}
A = \begin{bmatrix}
a_{11} & a_{12} & a_{13} & a_{14}\\
a_{21} & a_{22} & a_{23} & a_{24}\\
a_{31} & a_{32} & a_{33} & a_{34}\\
a_{41} & a_{42} & a_{43} & a_{44}
\end{bmatrix}.  
\end{equation*} 

Det har dock visats att andra typer av modeller i vissa fall presterar
bättre än den ergodiska. Elmezain och hans kollegor (Elmezain08)
presenterar t.ex resultat som pekar ut Bakis-modellen, även kallad
left-right-modellen, som den främsta då det gäller att modellera
tidsvarierande signaler. Bakis-modellen fungerar på så sätt att den
inte tillåter förflyttning till ett tidigare tillstånd. Elementen i
övergångsmatrisen A tvingas med andra ord efterfölja restriktionen

\begin{equation*}
a_{ij} = 0 \quad\text{om }j<i.
\end{equation*}

Vidare får sannolikhetsfördelningen för begynnelsetillstånd följande
utseende

\begin{equation*}
\pi_i = \begin{cases}
         1 & \quad\text{om } i = 1\\
         0 & \quad\text{om } i \neq 1.\end{cases}.
\end{equation*}  

Man startar alltså alltid i det första tillståndet för att sedan
stanna kvar i samma tillstånd eller röra sig åt höger. Antalet steg
man tar är inte begränsat på annat sätt än att det slutgiltiga
tillståndet $S_N$ inte får passeras. Denna modell användes av Yamato
och kollegor (Yamato92) och skiljer sig i vissa viktiga avseenden från
den som Elmezain använder. Elmezain08 tillåter bara förflyttning från
ett tillstånd till sig självt eller till nästa, ett val som ger upphov
till övergångsmatrisen

\begin{equation*}
A = \begin{bmatrix}
a_{11} & a_{12} & 0 & 0\\
0 & a_{22} & a_{23} & 0\\
0 & 0 & a_{33} & a_{34}\\
0 & 0 & 0 & a_{44}
\end{bmatrix}.  
\end{equation*} 

Utöver dessa varianter finns det självklart otaliga andra då man kan
sammanbinda de olika tillstånden helt efter eget tycke.  

\end{document}
