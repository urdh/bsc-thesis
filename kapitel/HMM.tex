\documentclass[../rapport_MVEX01-11-05]{subfiles}
\begin{document}

\subsection{Dolda Markovmodeller (HMM)}\label{sec:HMM}
För att kunna klassificera dynamiska gester räcker det inte att 
analysera bilder var för sig. Man behöver en modell som kan beskriva
karakterstiska beteenden för sekvenser av bilder.  

\notes{Förklara eller hänvisa till Markovmodeller/Markovteori också!}
En dold Markovmodell eller Hidden Markov model (HMM) är en stokastisk 
signalmodell som
beskriver ett system bestående av ett ändligt antal tillstånd. Likt en
vanlig Markovmodell förflyttar man sig sedan efter en given
sannolikhetsfördelning mellan tillstånden. Det som är dolt i en HMM är
det faktum att varje tillstånd i sin tur genererar
symboler efter ännu en sannolikhetsfördelning. Där man i standardfallet
observerar modellens tillstånd direkt, observaras nu istället de
symboler som genereras då tillstånden växlar, symboler som motsvarar
den verkliga process man önskar modellera. Ett förtydligande exempel
lyder som följer: 

\begin{quote}
En person har ett antal tärningar, vissa är viktade och avviker därmed
från den annars uniforma fördelningen. Han kastar en tärning ur sin
samling och berättar för en kompis hur många ögon tärningen
visar. Därefter avgör tärningskastaren om han vill byta tärning eller
inte, och upprepar proceduren. Kompisen får alltså inte se vilken av
tärningarna som kastas för tillfället, utan får bara ta del av vad
respektive tärning visar. Denna process skulle kunna beskrivas med
hjälp av en HMM där varje enskild tärning representerar ett
tillstånd. Den förstnämnda personen rör sig mellan tärningarna
(tillstånden) efter eget behag och tillkännager enbart värdet
(symbolerna) som genereras av dem. 
\end{quote}

Problematiken består då i att konstruera den HMM som bäst beskriver
processen. För klassificering är problemet sedan att välja den troligaste
modellen
för observerade data. \notes{FÖRSLAG: För just identifiering av
gester kan varje gest tänkas representeras av en
modell, där varje tillstånd motsvarar ett skede i gesten och alla möjliga kombinationer av egenskaper motsvaras av observationerna.
Observationerna tilldelas exempelvis genom någon typ av kodbok för egenskapsrummet.}

\notes{Ska flyttas till bakgrund. HMM tillämpat på gestigenkänning och det
relaterade problemet röstigenkänning är inget nytt. Redan i
slutet av 60-talet presenterade Baum \cite{Baum66,Baum67,Baum68,Baum70,Baum72} den
grundläggande teorin, och i början av 70-talet
implementerades modellen i samband med röstigenkänning av både
\citeasnoun{Baker75} och \citeasnoun{Jelinek69}. I slutet av
80-talet presenterades en sammanfattande artikel om röstigenkänning
med hjälp av HMM \cite{Rabiner89}, som till stor del ligger till
grund för följande teoretiska framställning. Tillämpningar gentemot
gestigenkänning gjordes bland annat i början av 90-talet av \citeasnoun{Yamato92},
och så sent som 2008 av \citeasnoun{Elmezain08}.}
% (Kan även nämna Starner98, Rigol98)}

\subsubsection{Beståndsdelarna av en HMM}
En HMM karakteriseras av följande:
\begin{itemize}
\item Antalet tillstånd i modellen $N$. Vi betecknar de olika
  tillstånden som $S = \{S_1, S_2, \dots, S_N\}$, och tillståndet vid
  tiden $t$ som $q_t$.
\item Antalet distinkta observationssymboler $M$, för ett givet
  tillstånd. De individuella symbolerna betecknar vi som $V =
  \{v_1,v_2,\dots,v_M\}$.
\item En övergångsmatris $A$, där varje element ger sannolikheten att
  vandra från ett tillstånd till ett annat, 
\begin{equation*}
a_{ij} = \Prob(q_{t+1} = S_j|q_t = S_i),\quad 1 \leq i,j \leq N.
\end{equation*}
I fallet då man kan nå alla tillstånd från vilket tillstånd som helst
med ett enda steg har vi alltså $a_{ij} > 0,\;\forall i,j$. 
\item Sannolikhetsfördelningar $B = b_j(k)$ för observationssymboler givet ett
  specifikt tillstånd $j$, där 
\begin{align*}
b_j(k) = \Prob(v_k \text{ vid tiden } t|q_t = S_j) \quad &1 \leq j \leq N\\
&1 \leq k \leq M.
\end{align*}
$B(j,k)$ ger med andra ord sannolikheten att symbolen $v_k$ genereras
då vi befinner oss i tillstånd $j$.
\item Sannolikhetsfördelningen för ett begynnelsetillstånd, $\pi =
  \{\pi_i\}$, där
\begin{equation*}
\pi_i = \Prob(q_1 = S_i).
\end{equation*}
\end{itemize}
Givet parametrarna $M$ och $N$, en specificering av symboler, samt de
tre sannolikhetsmåtten $A, B$ och $\pi$ är det% alltså
möjligt att
fullständigt specificera en HMM. I fortsättningen kommer vi att
använda oss av den kompaktare notationen $\lambda = (A,B,\pi)$
för att representera parametrarna för en given HMM.

\subsubsection{Grundläggande problem}
För att kunna föra över modellen från teori till praktisk tillämpning
återstår flera viktiga detaljer. \citeasnoun{Rabiner89} presenterar
tre grundläggande frågor för implementateringen av en HMM, varav följande
två är av intresse i den här rapporten:
\begin{description}
\item[Problem 1:] Givet en sekvens av observationer
  $\vect{O}=\{O_1,O_2,\dots,O_t\}$ och en modell $\lambda = (A,B,\pi)$, hur
  beräknar man $\Prob(\vect{O}|\lambda)$? Dvs. vad är sannolikheten att
  sekvensen genererades av modellen $\lambda$? Hur kan
  det göras så effektivt som möjligt? Effektivitetsfrågan visar sig
  vara ytterst viktig. 
\item[Problem 2:] Hur går man tillväga för att optimera
  $\Prob(\vect{O}|\lambda)$ genom att justera
  modellparametrarna $\lambda = (A,B,\lambda)$? 
\end{description}

Kan man lösa det första problemet har man möjlighet att uppskatta hur
väl en modell matchar en viss observationsföljd. Man kan se det som
ett sätt att jämföra olika modeller och
deras relation till en specifik observationsföljd. Denna synvinkel är
mycket användbar då man ställs inför utmaningen att välja mellan
flera existerande modeller. 

Det andra problemet behandlar modelloptimering. Hur kan man med en
grundmodell och en observationsföljd optimera modellen på så sätt att
sannolikheten att den givna observationsföljden genereras blir
maximal? Lösningen till detta problem utgör en fundamental del av
vägen till att skapa modellen i 
praktiken. Med hjälp av en träningssekvens, alltså en följd av
observationer, kan man då träna en HMM. Vikten av att
optimeringsprocessen ger bra resultat är stor eftersom
träningssekvensen utgör kopplingen till det verkliga fenomen
man önskar modellera.      

\subsubsection{Framåt-bakåt-tekniken}
\notes{Använda bilder i denna sektion?}

En användbar teknik för att lösa det första problemet introducerades av \citeasnoun{Baum67}, kallad
\emph{forward-backward}-proceduren. Vi
definierar framåtvariabeln $\alpha_t(i)$ som 
\begin{equation*}
\alpha_t(i) = \Prob(O_1,O_2,\dots,O_t,q_t = S_i | \lambda),
\end{equation*}
alltså sannolikheten att ha genererat observationsföljden $\textbf{O}
= \{O_1,O_2,\dots,O_t$\}, samt befinna sig i tillstånd $S_i$ vid tiden
$t$, givet en modell $\lambda$. En av metodens fördelar är att man med
hjälp av induktion kan finna $\alpha_t(i)$. Processen utgörs av
följande tre steg: 

\begin{enumerate}
\item Initialisering:
\begin{equation*}
\alpha_1(i) = \pi_ib_i(O_1), \quad 1\leq i \leq N
\end{equation*}

\item Induktion:
\begin{align*}
\alpha_{t+1}(j) =
\left[~\sum_{i=1}^N\alpha_t(i)a_{ij}~\right]b_j(O_{t+1}) \quad &1 \leq t \leq T-1 \\
&1 \leq j \leq N
\end{align*}

\item Avslutning:
\begin{equation*}
\Prob(\vect{O}|\lambda) = \sum_{i=1}^N\alpha_T(i)
\end{equation*}
\end{enumerate}

I steg 1, då $t=1$, initialiseras framåtvariabeln $\alpha_1(i)$ som
sannolikheten att börja i tillstånd $S_i$ och därefter generera
observation $O_1$. För $t=2$ använder man induktionssteget. Utgår man
från steg 1 kan man övertyga sig om att
$\sum_{i=1}^N\alpha_1(i)a_{ij}$ måste vara den totala sannolikheten
att hamna i $S_j$ vid $t=2$, efter att ha genererat $O_1$ i föregående
tillstånd. Efter multiplikation med $B_j(O_{2})$ fås därefter
sannolikheten att generera $O_2$, då man befinner sig i $S_j$ med en
observerad symbol $O_1$, eller med andra ord, $\Prob(O_1,O_2,S_j
= q_2 | \lambda) = \alpha_2(j)$. Fortsätter man på samma vis har man
slutligen $\alpha_T(i) = \Prob(O_1,O_2,\dots,O_T,S_i = q_T |
\lambda)$. Steg 3 ger slutligen svaret på problem 1, nämligen
sannolikheten $\Prob(\vect{O}|\lambda)$. Detta inses lätt då vi
använder oss av definitionen av $\alpha_T(i)$, 
\begin{equation*}
\Prob(\vect{O}|\lambda) = \sum_{i=1}^N\alpha_T(i) =
\sum_{i=1}^N\Prob(O_1,O_2,\dots,O_T,q_T = S_i|\lambda). 
\end{equation*} 

Med andra ord är sannolikheten att generera observationsföljden
$\vect{O}$ summan av sannolikheterna att generera $\vect{O}$ med
sluttillstånd $S_i$, $1 \leq i \leq N$.

\subsubsection{Iterativ metod för träning av HMM}

Med en lösning till problem 1 i bagaget vänder vi oss till problem 2,
själva träningen av en HMM. Detta är ett betydligt mer invecklat
problem än det tidigare, med betydligt fler angreppsvinklar. En
analytisk lösning till optimeringsproblemet är inte känd, utan man
tvingas istället använda sig av Baum-Welch-metoden eller
gradient-tekniker \cite{Dempster77,Levinson83}. 
Givet en ändlig observationssekvens går det inte att optimera modellens parametrar globalt \cite{Rabiner89}.
Däremot kan man välja $\lambda = (A,B,\pi)$ på så sätt att $\Prob(\vect{O}|\lambda)$ är lokalt
maximerad. 

Baum-Welch-metoden fungerar som följer:
på liknande sätt som för framåtvariabeln $\alpha_t(i)$
definierar vi nu bakåtvariabeln $\beta_t(i)$: 
\begin{equation*}
\beta_t(i) = \Prob(O_{t+1},O_{t+2},\dots,O_T | q_t = S_i, \lambda).
\end{equation*} 

$\beta_t(i)$ är alltså sannolikheten för en delsekvens av
observationer då man startar i tillstånd $S_i$. Processen för att beräkna $\beta_t(i)$ består av två steg:
\begin{enumerate}
\item Initialisering: 
\begin{equation*}
\beta_T(i) = 1, \quad 1 \leq i \leq N.
\end{equation*}
\item Induktion: 
\begin{align*}
\beta_t(i) = \sum\limits_{j=1}^Na_{ij}b_j(O_{t+1})\beta_{t+1}(j), \quad &t =
T-1,T-2,\dots,1 \\
&1 \leq i \leq N.
\end{align*}
\end{enumerate}  

\marginpar{okej med $\xi_t,\gamma_t$?}
Inledningsvis definieras $\beta_T(i)$ godtyckligt till $1$, varefter
induktionssteget fungerar efter samma princip som för framåtvariabeln
$\alpha$. Dessa variabler kan användas tillsammans
för att bilda två nya variabler $\xi_t$ och $\gamma_t$. Den första definieras som 
\begin{equation*}
\xi_t(i,j) = P(q_t = S_i, q_{t+1} = S_j|\vect{O},\lambda),
\end{equation*}
alltså sannolikheten att befinna sig i tillstånd $S_i$ och
därefter i $S_j$. Genom $\alpha$ och $\beta$ kan den uttryckas som
\begin{equation*}
\xi_t(i,j) = \frac{\alpha_t(i)a_{ij}b_j(O_{t+1})\beta_{t+1}(j)}{\Prob(\vect{O}|\lambda)}
\end{equation*} 
där $\Prob(\vect{O}|\lambda)$ gör att $\xi_t$ får ett korrekt
sannolikhetsmått.
Den andra variabeln kan uttryckas
\begin{equation*}
\gamma_t(i) = \sum_{j=1}^N\xi_t(i,j).
\end{equation*}

Summerar vi sedan $\gamma_t(i)$ över tiden $t$ får vi en storhet
som kan tolkas som antalet gånger tillståndet $S_i$ har besökts,
eller antalet förflyttningar från tillståndet $S_i$.
Med andra ord:
\begin{equation*}
\sum_{t=1}^{T-1}\gamma_t(i) = \text{antalet förväntade förflyttningar
  från tillstånd $S_i$.}
\end{equation*} 

På samma sätt kan summation över tiden hos 
$\xi_t(i,j)$ förklaras:
\begin{equation*}
\sum_{t=1}^{T-1}\xi_t(i,j) = \text{antalet förväntade förflyttningar
  från tillstånd $S_i$ till $S_j$}.
\end{equation*}
Då $\xi_t(i,j)$ inte är definierad för $t=T$ går summationen 
till $T -1$.

Den iterativa metoden som återuppskattar modellens parametrar $A,B$
och $\pi$ kan nu skrivas som
\begin{equation*}
\bar{\pi} = \text{[förväntat antal gånger i tillstånd $S_i$ vid
  $t=1$]} = \gamma_i(i),
\end{equation*}
\begin{multline*}
\bar{a}_{ij} = \\ = \frac{\text{förväntat antal förflyttningar från
    tillstånd $S_i$ till $S_j$}}{\text{förväntat antal förflyttningar
    från tillstånd $S_i$}} = \\ =
\frac{\sum_{t=1}^{T-1}\xi_t(i,j)}{\sum_{t=1}^{T-1}\gamma_t(i)},
\end{multline*}
\begin{multline*}
\bar{b}_j(k) = \\ = \frac{\text{förväntat antal gånger i tillstånd $S_j$
    och observerandes symbol $v_k$}}{\text{förväntat antal gånger i
    tillstånd $S_j$}} = \\ = \frac{\sum_{\substack{t=1\\s.t~ O_t =
      v_k}}^{T-1}\gamma_t{j}}{\sum_{t=1}^{T-1}\gamma_t(j)}.
\end{multline*}

Vänsterleden fås genom att ta den nuvarande modellen $\lambda =
(A,B,\pi)$, sedan beräkna högerleden, och därigenom erhålla $\bar{\lambda} =
(\bar{A},\bar{B}, \bar{\pi})$. \citeasnoun{Baum68} och \citeasnoun{Baker75}
visade att denna iterativa process resulterar antingen i
\begin{inparaenum}[\itshape1\upshape)]
	\item ett lokalt maximum för $\Prob(\vect{O}|\lambda)$
  (alltså att $\lambda = \bar{\lambda}$), eller
 	\item i en förbättrad modell $\bar{\lambda}$ på så sätt att
  $\Prob(\vect{O}|\bar{\lambda}) > \Prob(\vect{O}|\lambda)$
\end{inparaenum}. 

\subsubsection{Olika typer av HMM}
Då man talar om olika typer av HMM syftar man först och främst på
typen av övergångsmatris som används. Hittills har bara 
fallet där $a_{ij} > 0$ för alla $i,j$ nämnts --- detta kallas en ergodisk
modell. Detta innebär i sammanhanget att varje
tillstånd kan nås från alla andra tillstånd inom ändlig tid. Om man
tänker sig en modell med fyra tillstånd får man följaktligen övergångsmatrisen
\begin{equation*}
A = \begin{bmatrix}
a_{11} & a_{12} & a_{13} & a_{14}\\
a_{21} & a_{22} & a_{23} & a_{24}\\
a_{31} & a_{32} & a_{33} & a_{34}\\
a_{41} & a_{42} & a_{43} & a_{44}
\end{bmatrix}.  
\end{equation*} 

Det har dock visats att andra typer av modeller i vissa fall presterar
bättre än den ergodiska. \citeasnoun{Elmezain08}
presenterar resultat som pekar ut Bakis-modellen, även kallad
vänster-höger-modellen, som den främsta då det gäller modellering
av tidsvarierande signaler. Bakis-modellen fungerar på så sätt att den
inte tillåter förflyttning till ett tidigare tillstånd. Elementen i
övergångsmatrisen A tvingas med andra ord efterfölja restriktionen
\begin{equation*}
a_{ij} = 0 \quad\text{om }j<i.
\end{equation*}

Vidare får sannolikhetsfördelningen för begynnelsetillstånd följande
utseende
\begin{equation*}
\pi_i = \begin{cases}
         1 & \quad\text{om } i = 1\\
         0 & \quad\text{om } i \neq 1.\end{cases}.
\end{equation*}  

Man startar alltså alltid i det första tillståndet för att sedan
stanna kvar i samma tillstånd eller röra sig åt höger. Antalet steg
man tar är inte begränsat på annat sätt än att det slutgiltiga
tillståndet $S_N$ inte får passeras. Denna modell användes av
\citeasnoun{Yamato92} och skiljer sig i vissa avseenden från
den som den som används av Elmezain. \citeasnoun{Elmezain08} tillåter bara
förflyttning från
ett tillstånd till sig självt eller till nästa, ett val som ger upphov
till övergångsmatrisen
\begin{equation*}
A = \begin{bmatrix}
a_{11} & a_{12} & 0 & 0\\
0 & a_{22} & a_{23} & 0\\
0 & 0 & a_{33} & a_{34}\\
0 & 0 & 0 & a_{44}
\end{bmatrix}.  
\end{equation*} 

Utöver dessa varianter finns det självklart otaliga andra då man kan
sammanbinda de olika tillstånden helt efter eget tycke.  

\end{document}
