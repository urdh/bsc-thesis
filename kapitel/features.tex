\documentclass[../rapport_MVEX01-11-05]{subfiles}
\begin{document}
\subsection{Egenskaper}
En Hidden Markov Model behöver observationer för att fungera; dessa
observationer kommer vi att identifiera med hjälp av klassificering av
nyckelbilder ur videon utifrån ett antal egenskaper vi identifierar
med hjälp av den binära hudkartan för motsvarande bild.

\subsubsection{Hausdorffavstånd}

Hausdorffavståndet mellan den bild man klassificerar och ett antal
lagrade och förklassificerade gester kan användas som en egenskap för
klassificering \cite{Nielsen04}. Måttet mäter i någon mening avståndet
mellan två punktmängder $A$ och $B$ (i detta fallet mängderna som definieras av
randen till våra handområden) och definieras av

\begin{equation*}
  H(A, B) = \max\left(h(A,B),h(B, A)\right)
\end{equation*}

där

\begin{equation*}
  h(A, B) = \max\limits_{a\in A}\min\limits_{b\in
  B}\left|\left|a-b\right|\right|.
\end{equation*}

(fortsätt)

\subsubsection{Centroidhastighet, vinkel, konvexitet?}

\end{document} 

