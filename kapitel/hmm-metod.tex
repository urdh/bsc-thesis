\documentclass[../rapport_MVEX01-11-05]{subfiles}
\begin{document}

\subsection{Implementering av HMM för klassificering av icke-statiska gester}
För att klassificera icke-statiska gester bygger man vidare på metoden som användes i det 
statiska fallet. Steget till HMM kräver att gester utgörs av följder av index istället
för ensamma. I linje med HMM-teorin motsvarar dessa index de symboler som modellerna genererar, 
och i fortsättningen används endast ordet symbol. Idén är att representera varje gest, statisk som icke-statisk, i form av 
en symbolföljd. De statiska definieras lättast som två eller tre symboler av samma typ, medan 
de icke-statiska kommer bestå av en följd av olika symboler. Då de statiska gesterna redan tilldelats
symboler ligger problemet i att tilldela icke-statiska gester symboler på ett effektivt sätt. Vår tanke
var att dela upp filmerna med en icke-statisk gest i tre kategorier; prototypfilmer, träningsfilmer och 
testfilmer. Tilldelningen sker genom att varje prototypfilm delas upp i fyra lika stora delar. Den 
första fjärdedelen tilldelas en symbol, den andra en annan osv. Förhoppningen är att liknande 
delgester (behöver omformuleras) på så vis tilldelas samma symbol. 

Prototypgesterna placeras i kodboken. 

Programmet kommer därefter omvandla träningsfilmerna till symboler via kodboken. Längden av 
symbolföljderna för respektive gest beror på programmets prestanda, dvs hur många bilder 
man önskar analysera per sekund. Det bör uppmärksammas att varianter av en och samma gest 
kommer, beroende på deras längd, representeras av olika långa symbolföljder, vilket leder till 
att vi tränar flera HMM-modeller för varje gest. En möjlighet är att gruppera in symbolsekvenserna 
i tre längder för att motverka alltför strikta krav på utförande. 

Vi använder Bakis-modellen. 
Detta medför att varje HMM måste tränas med flera observationsföljder. (Rabiner89)
Nämna hur A och B initieras för bästa resultat? 
Antal tillstånd för modellen lika många som gesten den modellerar har symboler.
HMMs för statiska gester tränas för att känna igen sekvenser av samma symbol.
Med tränade modeller undersöker man sedan sannolikheten att en symbolföljd genererades av en viss modell.
Bäst modell vinner.
Behöver göra bilder/flowchart.  

 \end{document}

