\documentclass[../rapport_MVEX01-11-05]{subfiles}
\begin{document}
\subsection{Statiska gester}
När vi använder endast tre gester; tumme upp, Spock och V-tecknet,
blir träffresultatet mycket gott. Vi lät inlärningsmängden består av
tre bilder på varje gest från sju personer \marginpar{hur många i
  inlärmning? Harald, Meffe, David, Gurra, Banan, Uffe, Joppe?}, dvs
totalt 63 bilder. Med filmer av samma statiska gester som
testmängd träffar programmet rätt på nära 99 \% av bildrutorna, då
klassificering med KNN (k=XXX)\marginpar{vilket k?} på 17 normerade
features används. Detta
resultat gäller både då vi använder filmer av personer som ingår i
inlärningsmängden och då vi låter en person som inte ingår
inlärningsmängden filmas. Mer detaljerat resultat finns i tabell
\ref{tab:3resInl} och \ref{tab:3resEmma}.

\begin{table}
  \centering  
  \caption{???}
  \label{tab:3resInl}
  \begin{tabular}{lccc}
    \toprule
    Verklig \ Tolkad & 1 & 2 & 3 \\
    \midrule
    1&a11&a12&a13 \\
    2&a21&a22&a23 \\
    3&a31&a32&a33 \\
    \bottomrule
  \end{tabular}
\end{table}

\begin{table}
  \centering
  \caption{???}
  \label{tab:3resEmma}
  \begin{tabular}{lccc}
    \toprule
    Verklig \ Tolkad & 1 & 2 & 3 \\
    \midrule
    1&a11&a12&a13 \\
    2&a21&a22&a23 \\
    3&a31&a32&a33 \\
    \bottomrule
  \end{tabular}
\end{table}

\end{document}
