\documentclass[../rapport_MVEX01-11-05]{subfiles}
\begin{document}
\subsection{Statiska gester}
När vi använder endast tre gester; tumme upp, Spock och V-tecknet,
blir träffresultatet mycket gott. Vi lät inlärningsmängden består av
tre bilder på varje gest från sju personer \marginpar{hur många i
  inlärmning? Harald, Meffe, David, Gurra, Banan, Uffe, Joppe?}, dvs
totalt 63 bilder. Med filmer av samma statiska gester som
testmängd träffar programmet rätt på nära 99 \% av bildrutorna, då
klassificering med KNN (k=XXX)\marginpar{vilket k?} på 17 normerade
features används. Detta
resultat gäller både då vi använder filmer av personer som ingår i
inlärningsmängden och då vi låter en person som inte ingår
inlärningsmängden filmas. Mer detaljerat resultat finns i tabell
\ref{tab:3resInl} och \ref{tab:3resEmma}.

\begin{center}
\label{tab:3resInl}
  \begin{tabular}{| l || c | c | c | }
    \hline
    
   Verklig \ Tolkad & 1 & 2 & 3 \\ \hline \hline
   1&a11&a12&a13 \\ \hline
   2&a21&a22&a23 \\ \hline
   3&a31&a32&a33 \\ \hline
    \hline
  \end{tabular}
\end{center}

\begin{center}
\label{tab:3resEmma}
  \begin{tabular}{| l || c | c | c | }
    \hline
    
   Verklig \ Tolkad & 1 & 2 & 3 \\ \hline \hline
   1&a11&a12&a13 \\ \hline
   2&a21&a22&a23 \\ \hline
   3&a31&a32&a33 \\ \hline
    \hline
  \end{tabular}
\end{center}

\end{document}
