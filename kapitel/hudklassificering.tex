\documentclass[../rapport_MVEX01-11-05]{subfiles}
\begin{document}

\subsection{Klassificering av hudfärg}

För att skilja pixlar med hud från bakgrundspixlar, vilket är ett
centralt problem i projektet, används en modell där pixelfärgen antas
bero stokastiskt på om det är en hud- eller
bakgrundspixel. Pixelfärgen antas vara en multivariat normalfördelad
slumpvariabel, från någon av fem fördelningar; fyra för olika
hudfärger och en för icke-hud.

För att bestämma väntevärdesvektorn och kovariansmatrisen krävs ett
antal datapunkter, och till detta kan fördelningsparametrarna
anpassas med MLE. Den mulitvariata fördelningen ges av
\eqref{eq:hud:fordelning}, med kovariansmatris
$\Sigma$ och väntevärde $\mu$.

\marginpar{Snygga till!}

\begin{equation}
  \label{eq:hud:fordelning}
  p(x)=\frac{1}{(2\pi)^{d/2}|\Sigma|^{1/2}}e^{-(x-\mu)^T\Sigma^{-1}(x-\mu)/2}
\end{equation}

Givet $N$ datapunkter $x_i$ fås den logaritmerade sannolikheten $l(\mu,\Sigma|x_1,{\ldots} ,x_N)$:

\begin{equation}
  \label{eq:hud:logaritmerad}
  \begin{aligned}
  l(\mu,\Sigma|x_1,...,x_N) = &-\frac{Nd}{2}log(2\pi)-\frac{N}{2}log|\Sigma|\\
                              &-\sum_{i=1}^N(x_i-\mu)^T\Sigma^{-1}(x_i-\mu)/2
  \end{aligned}
\end{equation}

Vi får derivatan med avseende på $\mu$:

\begin{equation}
  \label{eq:hud:derivmu}
  \frac{\partial l}{\partial \mu}=\sum_{i=1}^N(x_i-\mu)^T\Sigma^{-1}
\end{equation}

{\ldots}och med avseende på $\Sigma^{-1}$ (efter en del algebra):

\begin{equation}
  \label{eq:hud:derivsigma}
  \frac{\partial l}{\partial \Sigma^{-1}}=\frac{N}{2}\Sigma -\sum_{i=1}^N(x_i-\mu)(x_i-\mu)^T/2
\end{equation}

Sätter vi dem i tur och ordning till noll får vi $\hat\mu$ och $\hat\Sigma$:

\begin{gather}
  \label{eq:hud:sigmamu}
  \hat\mu    =\frac{1}{N}-\sum_{i=1}^Nx_i\\
  \hat\Sigma =\frac{1}{N-1}\sum_{i=1}^N(x_i-\hat\mu)(x_i-\hat\mu)^T
\end{gather}

\citeasnoun{Elmezain08} studerade $18\:972$
hud-pixlar och $88\:320$ icke-hud-pixlar, och delade in dessa i fyra
hudfördelningar och en icke-hud-fördelning. De fyra hudfördelningarna
optimerades genom att varje pixel sorterades till en
ursprungsfördelning, varefter parametrarna uppdaterades, och alla
pixlar sorterades om efter de uppkomna fördelningarna --- detta gjordes
iterativt till konvergens. Deras ursprungsfördelningar har använts
som en grund för att identifiera hudfärg i denna rapport.

%Sådana beräkningar gjordes av \citeasnoun{Elmezain08} på 18972
%hud-pixlar och 88320 icke-hud-pixlar. Deras resultat har vi sedan
%använt i vårt arbete.

\subsubsection{SPM --- Skin Probability Map}

Förklara?

\end{document} 
