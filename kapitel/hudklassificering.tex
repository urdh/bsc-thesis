\documentclass[../rapport_MVEX01-11-05]{subfiles}
\begin{document}

\subsection{Klassificering av hudfärg och urskiljning av händer}\label{sec:hudklassificering}
Det första problemet vid visuell gestigenkänning är att kunna se handen.
För att göra detta
måste bilden först behandlas. Bildpunkterna klassificeras som hud eller
icke-hud. Som nästa problem ska bildområdet med hud i sin tur klassificeras som
en hand eller som övrig hud.

För att kunna isolera hud i en bild börjar man fördelaktigen med
att transformera bilden till en
lämplig färgrymd. Statistiska metoder kan sedan användas för att
markera bildpunkterna som hud respektive icke-hud.
Det andra problemet, att identifiera handen bland hudobjekten i bilden kan göras
exempelvis genom subtraktion av objekt som matchar ett ansiktes form. En enklare
metod är att ta det objekt som ligger längst till vänster i bilden, vilket
förklaras mer utförligt i \ref{sec:metod_hud:urskiljning}.

\subsubsection{Färgrymder}\label{sec:klassificering:fargrymder}
En färg kan matematiskt beskrivas som en vektor i ett färgrum, vars
basvektorer är som grundfärgerna i ett färgsystem.
För att kunna representera alla färger
behöver man ett tredimensionellt färgrum, men i vissa situationer kan
det räcka --- kanske till och med vara fördelaktigt --- med ett
tvådimensionellt färgrum \cite{Kakumanu07}.

Vilken färgrymd som är lämpligast för hudklassificering har inget
entydigt svar.
RGB-färgrymden (röd, grön, blå) har använts av bland annat
\citeasnoun{Lockton02} och \citeasnoun{Sebe04}, och ter sig lämplig
eftersom det är denna färgrymd som används i datorer. Perceptuella
färgrymder, såsom HSL, separerar kulör, mättnad och ljusintensitet.
De har inte rönt någon större framgång dels då deras definierande
transformationer innehåller singulariteter, dels då de är
prestandamässigt ofördelaktiga.

Ortogonala färgrymder har däremot nått utbredd användning för
hudklassificering \cite{Hsu02,Elmezain08,Hassanpour08} eftersom de
både separerar kulör och mättnad från ljusstyrka, och är affina avbildningar av
RGB. 
En ortogonal färgrymd som har visat sig särskiljt bra är $YC_bC_r$ \cite{Kakumanu07}.

\subsubsection[Färgrymden $\mathrm{YC_bC_r}$]{Färgrymden $\mathbf{YC_bC_r}$}
$YC_bC_r$-rymden är en ortogonalisering av RGB-rymden genom en
affin transformation, vilken resulterar i en basvektor som
representerar ljusstyrka ($Y$) och två som representerar färg (chroma), $C_b$ och $C_r$. Enligt
internationell standard \cite{ITU-BT601} ges
transformationen av
\begin{equation*}
  \label{eq:farg:ycbcr}
  \begin{gathered}
  Y   = 16  + ( 65.481R + 128.553G + 24.966B)\\
  C_b = 128 + (-37.797R - 74.203G  + 112.0B )\\
  C_r = 128 + (112.0R   - 93.786G  - 18.214B)
  \end{gathered}
\end{equation*}
där $R$, $G$, och $B$ är värden mellan $0$ och $255$ som representerar
färgens värde i RGB-rymden.

\subsubsection{Sannolikhetsfördelning för hudfärg}\label{sec:klassificering:hud}

För att skilja bildpunkter med hud från bakgrunden,
kan flera olika statistiska metoder
appliceras. Man har till exempel gjort försök med skogar av
slumpmässiga träd
\cite{Khan10}, Bayesiska nätverk \cite{Sebe04}, generativa modeller
\cite{Kruppa02}
och luddig aritmetik \cite{Shang10} --- en större undersökning av metoder
för huddetektion gjordes av \citeasnoun{Kakumanu07}.

En enklare metod som nått relativt stor framgång är den som baseras på
Gaussian Mixture Models \cite{Elmezain08,Hassanpour08}, det vill säga
en modell där bildpunktens färg antas bero stokastiskt på om
bildpunkten föreställer hud eller inte.

Färgen antas då vara en multivariat normalfördelad
stokastisk variabel, där fördelningens parametrar beror på om
pixeln motsvarar hud eller inte. För att kunna hantera olika hudfärger kan en
uppsättning av olika
fördelningar för respektive hudfärg användas. Den totala
sannolikhetsfördelning är sedan en linjärkombination av dessa. Detsamma
kan göras med sannolikhetsfördelningen för bildpunkter som inte
innehåller hud.

För att bestämma väntevärdesvektorn och kovariansmatrisen i en
uppskattad normalfördelning krävs ett
antal datapunkter, med vilkas hjälp parametrarna kan skattas med maximum
likelihood-metoden.

Fördelningen ges av
\begin{equation*}
  p(x)=\frac{1}{(2\pi)^{d/2}|\Sigma|^{1/2}}e^{-(\vect{x}-\mu)^T\Sigma^{-1}(\vect{x}-\mu)/2}
\end{equation*}
med kovariansmatris $\Sigma$ och väntevärde $\mu$.
Givet $N$ datapunkter $\vect{x}_i,\:1\leq i\leq N$ ges den logaritmerade
sannolikheten $l(\mu,\Sigma|\vect{x}_1,{\ldots} ,\vect{x}_N)$ av
\begin{equation*}
  \begin{aligned}
  l(\mu,\Sigma|\vect{x}_1,...,\vect{x}_N) = &-\frac{Nd}{2}\log(2\pi)-\frac{N}{2}\log|\Sigma|\\
                              &-\sum_{i=1}^N\frac{(\vect{x}_i-\mu)^T\Sigma^{-1}(\vect{x}_i-\mu)}{2}.
  \end{aligned}
\end{equation*}

Vi kan därefter beräkna derivatan med avseende på $\mu$,
\begin{equation*}
  \frac{\partial l}{\partial \mu}=\sum_{i=1}^N(\vect{x}_i-\mu)^T\Sigma^{-1},
\end{equation*}
samt derivatan med avseende på $\Sigma^{-1}$,
\begin{equation*}
  \frac{\partial l}{\partial \Sigma^{-1}}=\frac{N}{2}\Sigma
  -\sum_{i=1}^N\frac{(\vect{x}_i-\mu)(\vect{x}_i-\mu)^T}{2}.
\end{equation*}

Sätter man dem i tur och ordning till noll fås uppskattningarna
$\hat\mu$ och $\hat\Sigma$:
\notes{Hur då? Det är \emph{inte} uppenbart!}
\begin{gather*}
  \hat\mu    =\frac{1}{N}\sum_{i=1}^N\vect{x}_i\\
  \hat\Sigma =\frac{1}{N-1}\sum_{i=1}^N(\vect{x}_i-\hat\mu)(\vect{x}_i-\hat\mu)^T
\end{gather*}

För att ytterligare förbättra fördelningarna kan de antas vara
en linjärkombination av flera normalfördelningar. En iterativ metod för optimering av
sådana fördelningar beskrivs av \citeasnoun{Elmezain08}.

För varje bildpunktsfärg kan med dessa fördelningar sannolikheten för att färgen ska
förekomma då bildpunkten föreställer hud
respektive ickehud bestämmas. Det är rimligt att klassa en pixel som
hud om det gäller att
$\Prob(\textrm{hud}|\textrm{färg})>\Prob(\textrm{ickehud}|\textrm{färg})$
. Det innebär att om kvoten
\begin{multline*}
\frac{\Prob(\textrm{hud}|\textrm{färg})}{\Prob(\textrm{ickehud}|\textrm{färg})}=\\
=\frac{\Prob(\textrm{färg}|\textrm{hud})\Prob(\textrm{hud})}{\Prob(\textrm{färg})}
 \frac{\Prob(\textrm{färg}}{\Prob(\textrm{ickehud})\Prob(\textrm{ickehud}|\textrm{färg})}=\\
=\frac{\Prob(\textrm{färg}|\textrm{hud})\Prob(\textrm{hud})}{\Prob(\textrm{färg}|\textrm{ickehud})
 P(\textrm{ickehud})}
\end{multline*}
är större än någon konstant (exempelvis ett), så sägs bildpunkten föreställa
hud. Med denna information kan man sedan sätta en tröskel på den behandlade
bilden, vilket resulterar i en binär bild som kan behandlas ytterligare
innan egenskaper tas fram utifrån handens form.

\subsubsection{Morfologiska operationer}\label{sec:morph}

Morfologiska operationer kan användas för att modifiera ''objekt'' i
binära bilder, till exempel för att rensa bort brus eller för att
mjuka upp kanter. Det finns fyra grundläggande morfologiska
operationer: \emph{erosion}, \emph{dilatation}, \emph{öppning} och
\emph{stängning} \cite[s.~25]{Rudemo09}.

Dessa definieras som operationer på mängder; specifikt mängden $A$
av alla svarta pixlar i bilden vi behandlar, dvs. alla pixlar som är
''avstängda'' och mängden $S$ som representerar ett strukturelement
som i någon mening förändrar hur operationen verkar på bilden.
Strukturelementet kan dessutom förflyttas så att dess referenspixel
ligger på plats $(i,j)$ i bilden --- kalla detta förflyttade
strukturelement för $S_{i,j}$.

\emph{Erosionen} eroderar bort delar av de svarta områdena genom
att endast behålla de pixlar vars ''omgivning'' (definierad av mängden
$S_{i,j}$) ligger helt i mängden $A$. Vad operationen egentligen gör är alltså
att skala av det yttersta lagret, randen, av mängden $A$. Erosion definieras enligt
\begin{equation*}
  A\ominus S = \{(i,j)\;:\;S_{i,j}\subseteq A\}.
\end{equation*}

Dess motsats, \emph{dilatation}, agerar som
en erosion på komplementet till $A$ och definieras enligt
\begin{equation*}
  A\oplus S = (A^C\ominus S)^C.
\end{equation*}

Dessa operationer är i sig inte särskilt användbara då de förändrar
arean av objektet i bilderna. Operationerna kan dock kombineras för
att skapa operationer som i någon mening endast 
''mjukar upp'' objektet i bilden utan att förändra deras area markant.
Eftersom erosion och dilatation är motsatser kan dessa appliceras i
följd för att behandla bilden och
sedan återställa den. Detta ger upphov till två nya operationer:
\emph{öppning} och \emph{stängning}.

En öppning är en erosion följd av en dilatation, enligt
\begin{equation*}
  \phi_S(A)=(A\ominus S)\oplus S'.
\end{equation*}

Detta resulterar i att svart ''brus'', dvs.~svarta områden som inte kan täckas
av $S$, försvinner i
första steget varefter större svarta områdena utökas till sin ursprungliga
storlek i det andra
steget. Operationen ''öppnar upp'' vita områden.

En stängning består av motsatt procedur, enligt
\begin{equation*}
  \Phi_S(A)=(A\oplus S)\ominus S',
\end{equation*}
och ger även motsatt resultat (svarta områden ''öppnas upp'').

De
båda operationerna använder utöver $A$ och $S$ även $S'$, vilket är
$S$ roterad 180\textdegree{} runt sin referenspixel.

\end{document} 
