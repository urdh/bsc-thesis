\documentclass[../rapport_MVEX01-11-05]{subfiles}
\begin{document}
Med de metoder som beskrivits nås alltså en träffsäkerhet på 91.4\,\%
för våra tio statiska gester. I ett antal stycken framöver diskuteras
vad som skulle kunna göras för att ytterligare förbättra
träffsäkerheten, utvidga programmets funktionalitet eller på övrigt
sätt gå vidare från resultatet.

\section{Hudigenkänning}
Bättre fördelning från större datamängd, k-kluster.
Manuellt skrapande i SCM

\section{Handidentifiering}

\section{\knn}
Något som kan göras för att förbättra \knn-metodens träffsäkerhet är först och
främst att utöka prototypmängden till fler punkter per gest. En annan möjlig
förbättring skulle vara att inspektera de bilder som faktiskt ligger bakom
punkten i egenskapsrummet. Om den uppenbart är felaktig, och kanske hade varit
svår att tolka som rätt gest även för en människa, kan den med fördel tas bort.
Slarvigt utförda gester kommer då mindre sannolikt att
klassificeras rätt.\notes{om alls? thresholding?!}

\section{Features}
Inre av handen (linjer etc)

Kurvatur (fingertoppar, avstånd däremellan)

\section{Realtid}
\section{Tillämpningar}
Alfabetsigenkänning, sten-sax-påse

\section{Dynamiska gester}





\end{document}
