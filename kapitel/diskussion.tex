\documentclass[../rapport_MVEX01-11-05]{subfiles}
\begin{document}
De metoder vi använt ger en träffsäkerhet på 91.4\,\% för de tio statiska
gester vi valt. Denna träffsäkerhet kan sannolikt förbättras på ett antal
olika sätt, och tillämpningarna av vårt resultat kan vara stora. Dessutom
kan metoden utökas och generaliseras för att även kunna identifiera
dynamiska gester. I detta kapitel ämnar vi därför lägga fram förslag på framtida
förbättringar, tillämpningar och utökningar av vår metod och vårt program,

\section{Hudigenkänning}
De sannolikhetsfördelningar som använts för att skilja hud från annat
har fungerat tillräckligt bra för att kunna skilja ut handen i en
relativt kontrollerad miljö. Bättre resultat skulle dock kunna uppnås
med bättre sannolikhetsfördelningar. Det man främst skulle kunna göra
är att använda $k$-klustring, d.v.s.~linjärkombinationer av flera
sannolikhetsfördelningar, även för ickehud. Bättre fördelningar skulle
antagligen även fås med en större datamängd, alltså fler
exempelbilder.

Man skulle också kunna använda någon annan klassificerare, såsom \knn,
vid skapandet av vår SPM. Ett problem med den metoden är att det
kräver att man kasserar data för att få lika stora prototypmängder,
eller att man viktar datapunkterna.

Efter det statistiska urvalet skulle man kunna redigera direkt i
hudfärgskartan för att få ett bättre resultat. Detta är dock väldigt
tidskrävande.

\section{Handidentifiering}
Handidentifieringen kan förbättras på ett antal punkter;
den metod vi använder (att söka ett tillräckligt stort objekt från höger i
bilden) är relativt ''dum''. Förutsätter man att de två objekt som kan finnas
i bild är ett ansikte och en hand (vilket oftast är fallet) så kan man
istället välja handen genom att först identifiera ansiktet, exempelvis genom
den metod \citeasnoun{Hsu02} använder, för att sedan ''välja bort'' detta
objekt (dvs.~ta bort det från bilden). Vad man har kvar bör då,om brus tagits
bort med en morfologisk operation eller genom att välja de två största
objekten i bilden enligt \citeasnoun{Nielsen04}, vara handen.

Man skulle dessutom kunna klippa bort handleden på det sätt
\citeasnoun{Deimel99} föreslår, och därmed öka träffsäkerheten --- att
handleden ibland syns och ibland skuggas är något som gör vissa av våra
egenskaper allt för osäkra. Dessutom skulle man då kunna släppa på kravet
att användaren har en långärmad tröja.

Till sist kan man då man identifierat handen utnyttja denna information och
klippa bort resten av bilden i nästa iteration. På så sätt behöver man bara
identifiera handobjektet en gång i bilden, och kan koncentrera beräkningen
i följande iterationer på att beräkna egenskaper och klassificera gesten.
Man skulle med fördel kunna hoppa över egenskapsberäkningen i denna första
iteration.

\section{\knn}
För att förbättra \knn-metodens träffsäkerhet kan man först och
främst att utöka prototypmängden till fler punkter per gest. En annan möjlig
förbättrna i egenskapsrummet; om en bild är uppenbart är felaktig, och
kanske hade varit
svår att tolka som rätt gest även för en människa,
kan den med fördel tas bort.
Slarvigt utförda gester kommer då mindre sannolikt att
klassificeras rätt.\notes{om alls? thresholding?!}

\section{Egenskaper}
De egenskaper vi använt för klassificering har visat sig relativt
effektiva, men det finns modifieringar och tillägg som skulle kunna
öka träffsäkerheten.
En sådan modifiering är att istället för fyrkantigheten $\Delta
x/\Delta y$ använda dess logaritm, $\ln(\Delta x/\Delta
y)$, eftersom kvotens värdemängd är $(0,\infty)$ och tar värdet 1 då
höjen och bredden är samma, vilket innebär att egenskapen får större
viktning då handen är bredare än hög än tvärtom. Detta problem
försvinner då kvoten logaritmeras. Även övriga egenskaper som
definieras genom kvoter kan logaritmeras på liknande sätt.

Det finns även nya klasser av egenskaper som vi inte arbetat med
alls. Med hjälp av studier av handens kurvatur bör fingertoppar kunna
identifieras --- antal, avstånd och relativt läge för dessa skulle vara
mycket intressanta egenskaper, och antagligen ge mycket bra resultat.
En annan intressant egenskapsklass är de som utnyttjar mönster i det
inre av handen. Detta skulle kunna göras
genom att efter man identifierat handen studera originalbilden i dess
inre, t.ex.~genom ett kantdetektionsfilter, och där plocka ut mängden
kanter eller riktning genom så kallad Houghtransform \cite[s.~23]{Rudemo09}.

\section{Realtid}
Målet med en applikation för gestigenkänning är givetvis att kunna tolka
gester i realtid, vilket kan vara svårt eftersom bildbehandlingen och
klassificeringen som används ofta är beräkningsintensiv. Vår implementation
är skriven i \MATLAB, och skulle troligtvis vara mycket snabbare om den
skrivits i ett lågnivåspråk som C istället, men kan trots det utföra
beräkningarna snabbt nog för att nästan vara realtid.

För att kunna känna igen statiska gester behövs betydligt färre bilder per
sekund än de 24 våra webbkameror gett oss. Därmed kan man utan skam slänga
bort majoriteten av bilderna och endast behandla fyra eller sex bilder per
sekund, vilket skulle göra att gestigenkänningen ser ut att vara i realtid.
Detta skulle givetvis medföra att statiska gester som visas i mindre tid
än intervallet mellan bilderna inte skulle kännas igen.

\notes{Mer?}

\section{Tillämpningar}

\section{Dynamiska gester}





\section{Slutsats}
\notes{skriv nåt käckt här?}
\end{document}
