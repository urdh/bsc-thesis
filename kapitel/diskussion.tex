\documentclass[../rapport_MVEX01-11-05]{subfiles}
\begin{document}
Med de metoder som beskrivits nås alltså en träffsäkerhet på 91.4\,\%
för våra tio statiska gester. I ett antal stycken framöver diskuteras
vad som skulle kunna göras för att ytterligare förbättra
träffsäkerheten, utvidga programmets funktionalitet eller på övrigt
sätt gå vidare från resultatet.

\section{Hudigenkänning}
De sannolikhetsfördelningar som använts för att skilja hud från annat
har fungerat tillräckligt bra för att kunna skilja ut handen i en
relativt kontrollerad miljö. Bättre resultat skulle dock kunna uppnås
med bättre sannolikhetsfördelningar. Det man främst skulle kunna göra
är att använda k-klustring, d.v.s.~linjärkombinationer av flera
sannolikhetsfördelningar, även för ickehud. Bättre fördelningar skulle
antagligen även fås med en större datamängd, alltså fler
exempelbilder.

Man skulle också kunna använda någon annan klassificerare, såsom \knn,
vid skapandet av vår SCM. En svårighet med det är att det skulle
kräva kassering av data för att få lika stora prototypmängder,
eller viktning av datapunkterna.

Efter det statistiska urvalet skulle man kunna redigera direkt i
hudfärgskartan för att få ett bättre resultat. Detta är dock väldigt
tidskrävande.

\section{Handidentifiering}

\section{\knn}
Något som kan göras för att förbättra \knn-metodens träffsäkerhet är först och
främst att utöka prototypmängden till fler punkter per gest. En annan möjlig
förbättring skulle vara att inspektera de bilder som faktiskt ligger bakom
punkten i egenskapsrummet. Om en bild är uppenbart är felaktig, och
kanske hade varit
svår att tolka som rätt gest även för en människa, kan den med fördel tas bort.
Slarvigt utförda gester kommer då mindre sannolikt att
klassificeras rätt.\notes{om alls? thresholding?!}

\section{Egenskaper}
De egenskaper vi använt för klassificering har visat sig relativt
effektiva, men det finns modifieringar och tillägg som skulle kunna
öka träffsäkerheten.
En sådan modifiering är att istället för fyrkantigheten $\Delta
x/\Delta y$ använda dess logaritm, $\text{ln}(\Delta x/\Delta
y)$. Detta då kvotens värdemängd är $(0,\inf)$ och den tar värdet 1 då
höjen och bredden är samma. Det innebär att egenskapen får större
viktning då handen är bredare än hög än tvärtom. Detta problem
försvinner då kvoten logaritmeras. Även övriga egenskaper som
definieras genom kvoter kan logaritmeras på liknande sätt.

Det finns även nya klasser av egenskaper som vi inte arbetat med
alls. Med hjälp av studier av handens kurvatur bör fingertoppar kunna
identifieras. Antal, avstånd och relativt läge för dessa skulle vara
väldigt intressanta, och antagligen mycket karakteristiska.

En annan intressant egenskapsklass är de som utnyttjar mönster i det
inre av handen. Detta skulle kunna göras
genom att efter man identifierat handen studera originalbilden i dess
inre, t.ex.~genom ett kantdetektionsfilter, och där plocka ut mängden
kanter eller riktning genom s.k.~Houghtransform\notes{Har vi källa på
  det här?}

\section{Realtid}
\section{Tillämpningar}
\section{Dynamiska gester}





\section{Slutsats}
\notes{skriv nåt käckt här?}
\end{document}
