\documentclass[../rapport_MVEX01-11-05]{subfiles}
\begin{document}

\subsection{Bildanalys}


\subsubsection{Filter}


\subsubsection{Transformer}


\subsubsection{Morfologiska operationer}

\marginpar{Relevans?}

Morfologiska operationer kan användas för att modifiera ''objekt'' i
binära bilder, till exempel för att rensa bort brus eller för att
mjuka upp kanter. Det finns fyra grundläggande morfologiska
operationer: \emph{erosion}, \emph{dilatation}, \emph{öppning} och
\emph{stängning} \cite[ss.~25]{Rudemo09}.

Dessa kan definieras som operationer på mängder; specifikt mängden $A$
av alla svarta pixlar i bilden vi behandlar, dvs. alla pixlar som är
''avstängda'' och mängden $S$ som representerar ett strukturelement
som i någon mening förändrar hur operationen verkar på bilden.
Strukturelementet kan dessutom förflyttas så att dess referenspixel
ligger på plats $(i,j)$ i bilden --- vi kan kalla detta förflyttade
strukturelement för $S_{i,j}$.

\emph{Erosionen}, som eroderar bort delar av de svarta områdena genom
att endast behålla de pixlar vars ''omgivning'' (definierad av mängden
$S_{i,j}$) ligger helt i mängden $A$. Vad vi egentligen gör är alltså
att skala av det yttersta lagret, randen, av mängden $A$. Vi kan
enkelt definiera detta i termer av mängdlära, se \eqref{morf1}. Det
inses lätt att den motsatta operationen, \emph{dilatation}, agerar som
en erosion på komplementet till $A$ vilket kan definieras enligt
\eqref{morf2}.

\begin{equation}
  \label{morf1}
  A\ominus S = \{(i,j)\;:\;S_{i,j}\subseteq A\}
\end{equation}

\begin{equation}
  \label{morf2}
  A\oplus S = (A^C\ominus S)^C
\end{equation}

Dessa operationer är i sig inte särskilt användbara då de förändrar
storleken på objekten vi behandlar. Vi kan dock kombinera dessa två
operationer för att skapa operationer som i någon mening endast 
''mjukar upp'' objekten i bilden utan att förändra deras storlek på
ett markant sätt. Eftersom erosion och dilatation är motsatser kan vi
kombinera dessa (applicera dem i följd) för att behandla bilden och
sedan återställa den. Detta kan vi göra på två olika sätt, vilket
resulterar i två nya operationer: \emph{öppning} och \emph{stängning}.

En öppning är en erosion följd av en dilatation, enligt \eqref{morf3}.
Detta resulterar i att svart ''brus'', dvs. ensamliggande pixlar som
är svarta (eller små områden som inte kan täckas av $S$), försvinner i
första steget varefter de svarta områdena utökas igen i det andra
steget. Man kan säga att operationen ''öppnar upp'' vita områden.

En stängning består av motsatt procedur, enligt \eqref{morf4}, och
resulterar i princip i en ''öppning'' av de \emph{svarta} områdena. De
båda operationerna använder utöver $A$ och $S$ även $S'$, vilket är
$S$ roterad 180\textdegree{} runt sin referenspixel.
\marginpar{Men varför rotera med 180\textdegree?}

\begin{equation}
  \label{morf3}
  \phi_S(A)=(A\ominus S)\oplus S'
\end{equation}

\begin{equation}
  \label{morf4}
  \Phi_S(A)=(A\oplus S)\ominus S'
\end{equation}

Utöver dessa fyra operationer kan vi även \emph{yttre} och \emph{inre}
kanter till de svarta områdena i bilden. Med hjälp av erosion och
dilatation kombinerat med mängddifferensen $A\setminus B=\left\{x\in
A\;:\;x\not\in B\right\}$ kan vi definiera den yttre randen
$\partial_S^o A$ och den inre randen $\partial_S^i A$, båda med
avseende till strukturelementet $S$, enligt \eqref{morf5} och
\eqref{morf6}.

\begin{equation}
  \label{morf5}
  \partial_S^o A = \left(A\oplus S\right)\setminus A
\end{equation}

\begin{equation}
  \label{morf6}
  \partial_S^i A = A\setminus\left(A\ominus S\right)
\end{equation}

Operationerna är relativt enkla att implementera i MATLAB:

\begin{lstlisting}
% Erosion: erosion(A,S) motsvarar A \ominus S
erosion = @(A,S)(filter2(S/sum(S(:)), A) ~= 0);
% Dilatation: dilation(A,S) motsvarar A \oplus S
dilation = @(A,S)(~filter2(S/sum(S(:)), ~A) ~= 0);
% Öppning: opening(A,S) motsvarar \phi_S(A)
opening = @(A,S)(dilation(erosion(A,S), rot90(S,2)) ~= 0);
% Stängning: closing(A,S) motsvarar \Phi_S(A)
closing = @(A,S)(erosion(dilation(A,S), rot90(S,2)) ~= 0);

%% Yttre rand
%% Inre rand
\end{lstlisting}

\marginpar{Bättre formatering behövs till koden.}

\end{document} 
