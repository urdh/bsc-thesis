\documentclass[../rapport_MVEX01-11-05]{subfiles}
\begin{document}
\section{Avgränsning}

Även om grundproblemet i korthet kan formuleras som en mycket generell
gestigenkänning har denna rapport begränsats något. Först och främst
begränsas datamängden till enkel videobaserad indata, dvs.~inga sensorhandskar
eller stereoskopiska kameror eftersom detta gör implementeringen mer
tillgänglig. Dessutom koncentreras rapporten på träningsbara
klassificerare. Till sist har metoder baserade på modeller av virtuella händer
undvikits.
%(både 3d- och 2d-baserade) i form av siluetter, konturer eller liknande undvikits.

%\notes{jag ändrade lite här /Meffe, resten är utkommenterat..}
%För att kunna fokusera innehållet på funktionaliteten begränsas även
%implementateringen till ett programspråk vi känner oss hemma i (\MATLAB), vilket gör att
%tillämpbara resultat både i form av tillgänglighet och prestanda inte kommer
%att ha fokus. Istället satsar rapporten på att identifiera bra metoder mätt
%främst i andelen felklassificering.


\end{document} 
