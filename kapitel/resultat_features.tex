\documentclass[../rapport_MVEX01-11-05]{subfiles}
\begin{document}

\subsubsection{Egenskaper}\label{sec:resultat_features}

Egenskaperna vi använt har visat sig vara mycket bra, upp till en viss gräns.
Figur~\ref{fig:knn-optimering} visar att optimering av vilka egenskaper som
inkluderas ger ett distinkt minimum redan vid tio egenskaper --- dessa tio
bästa listas i tabell~\ref{tab:bestfeats}.

\begin{table}[tb]
	\centering
	\caption{De tio bästa egenskaperna}
	\label{tab:bestfeats}
	\begin{tabular}{ll}
		\toprule
		Ranking & Egenskap \\
		\midrule
		1 & Hu-moment 3 \\
		2 & Hu-moment 2 \\
		3 & Centroidläge, Y-led \\
		4 & Fyrkantighet \\
		5 & Soliditet \\
		6 & Excentricitet \\
		7 & Konvexitet \\
		8 & Hu-moment 1 \\
		9 & Utsträckning \\
		10 & Hu-moment 7 \\
		\bottomrule
	\end{tabular}
\end{table}

%% Egenskapsmetod
%Egenskaper kan förstås variera i kvalitet och det är därför viktigt att
%verifiera att de egenskaper som används verkligen gör det möjligt för
%klassificeringsalgoritmen att skilja på olika gester. Detta innebär att en
%gest inte kan variera mycket inom en gest --- för en enskild gest ska 
%egenskapen alltså endast anta värden på ett delintervall till värdemängden ---
%och att en egenskap gärna även ska variera mycket mellan gester, dvs.~att
%delintervallen inte överlappar.

%Det är viktigt för \knn-metoden att
%egenskapsrummet är utformat så att bilder från varje gest är både tydligt
%separerade och tydligt grupperade. Det viktigaste är grupperingen; att gester
%är separerade är något som endast förenklar processen.
%Trots att gesterna ligger mycket tätt i vårt egenskapsrum så är de redan i det
%tvådimensionella underrummet till egenskapsrummet som visas i figur~\ref{fig:feats1011}
%mycket tydligt grupperade, vilket gör att \knn-metoden ger exakta resultat.


%% kNN-resultat?
%Fler egenskaper än de två i figur~\ref{fig:feats1011} ger inte helt oväntat
%(men inte heller trivialt) ett bättre resultat. Figur~\ref{fig:knn-optimering} visar
%hur det relativa felet minskar både då $k$ ökar i \knn-metoden och då antalet
%egenskaper som inkluderas i rummet ökar. Något mer oväntat är dock att det
%finns ett tydligt minimum som inte inkluderar alla egenskaper.
%
%Det är även intressant att veta vilka av egenskaperna som är ''bäst'' när det
%gäller att klassificera gester. Tabell~\ref{tab:bestfeats} listar de tio bästa
%egenskaperna i den ordning \notes{vårt skript? sorterar dem?}

\end{document}
