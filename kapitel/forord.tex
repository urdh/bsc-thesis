\documentclass[../rapport_MVEX01-11-05]{subfiles}
\begin{document}

\notes{Riktigt förord?}

\subsection*{Tack till}
Vi skulle vilja tacka följande personer och organisationer för deras
ovärderliga hjälp i arbetet:

\begin{list}{}{\leftmargin=1em\rightmargin=1em}
\item \emph{Mats Rudemo} och \emph{Magnus Röding} för stöd och tips i arbetet
och skrivprocessen, och
\emph{Hans Malmström} för värdefulla synpunkter på den språkliga utformningen.

\item Våra tålmodiga försökspersoner \emph{Emma Kjelsson, Anna Nilsson,
Jonatan Rydberg, Dündar Göç, Susanne Schilliger Kildal, Gustav Hansson,
Fredrik Johnsson, Josefin Lövmark, Jakob Friman} och \emph{Christoffer
Johansson} för deras hjälpande händer.

\item Versionshanteringstjänsterna \emph{Bitbucket} och \emph{Mercurial} utan vilka
arbetet skulle befinna sig i fullständigt kaos.
\end{list}

Tack även till stöttande vänner och givmilda vädergudar.

%\subsection*{Bildrättigheter}
%Figur \ref{fig:knn-overview}: \copyright Antti Ajanki, 2007.\\

    
%I denna skall det anges vilka delar som skall tillskrivas respektive
%författare. Där skall också anges att en loggbok förts över de
%enskilda medverkandes prestationer under arbetet.
%\marginpar{+ kommentar om att vi kanske har vävt ihop våra bitar med varandra
%sedan\ldots (vilket vi bör göra snyggt)}
%Enligt reglerna får examination av examensarbete med flera författare
%inte ske utan att loggbok\footnote{Med loggbok menas här gruppens
%dagbok och sammanställning av de individuella tidsloggarna} inlämnats
%och rapportens inledning uppfyller villkoren ovan.

%\subsection*{Prestationsredovisning}
%Vi har bla bla bla
%
%Harald Freij
%
%Viktor Nilsson har skrivit den ursprunliga texten i följande kapitel:
%\ref{sec:knn}, \ref{sec:inledning}
%
%David Samuelsson
%
%Simon Sigurdhsson \ref{sec:features}, \ref{sec:resultat_features},
%\ref{sec:inledning}, \ref{sec:klassificering:fargrymd},
%\ref{sec:klassificering:ycbcr}, \ref{sec:klassificering:morfologi}, %\ref{sec:klassificering}

\end{document} 
%TODO: mer todo
