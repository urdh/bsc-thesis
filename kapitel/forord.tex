\documentclass[../rapport_MVEX01-11-05]{subfiles}
\begin{document}
Vid skrivandet av denna rapport har alla fyra författare 
med likvärdiga insatser redigerat och sammanfogat
texten till en helhet. Även om varje text ursprunligen har skrivits av en
person har alla haft betydande medverkan i alla avsnitt.
Vi vill ändå nämna vem som ligger bakom några av de större avsnitten.
De inledande delarna om färgrymder, avsnittet om morfologiska operationer
och genomgången av de olika egenskaperna skrevs först av Simon Sigurdhsson.
Sannolikhetsförderlningar för hudfärg, isolering av handen, samt våra gester
har främst förklarats av Harald Freij.
Delarna om dolda Markovmodeller kan så tillskrivas David Samuelsson, medan
delarna om \knn samt datainsamling främst skrivits av Viktor Nilsson.
Slutligen vill vi poängtera att alla har bidragit med texter varje kapitel.

Rapporten är skriven på Svenska, men då ämnet förutsätter ett flitigt nyttjande
av fackterminologi har vi i vissa fall tvingats välja engelskspråkiga termer.
Detta har vi endast gjort i de fall då en svensk översättning uppenbart hade
varit svår att förstå.

Under arbetets gång har en dagbok med detaljer kring arbetsprocessen
förts. Denna ligger delvis till grund för betygssättningen i
kandidatarbeteskursen. Den innehåller samtidigt information om sidospår som ej
tagits upp i rapporten, och kan tillsammans med versionshanteringsloggar ge
en inblick i hur mycket tid olika delar av arbetet har tagit samt vem som har
arbetat mest med vad. I stort kan sägas att ingen har haft några specifika
ansvarsområden förutom att Simon Sigurdhsson tillsammans med Viktor
Nilsson haft huvudansvaret för versionshantering och \LaTeX-relaterade frågor.
Alla har precis som i rapporten varit lika delaktiga i programmeringen.
%, och att Harald Freij har varit den som sammanställt ovan nämnda dagbok.

%\cleardoublepage
\section*{Tack till}
Vi skulle vilja tacka följande personer och organisationer för deras
ovärderliga hjälp i arbetet:

\begin{list}{}{\leftmargin=1em\rightmargin=1em}
\item \emph{Mats Rudemo} och \emph{Magnus Röding} för stöd och tips i arbetet
och skrivprocessen, och
\emph{Hans Malmström} för värdefulla synpunkter på den språkliga utformningen.

\item Våra tålmodiga försökspersoner \emph{Emma Kjelsson, Anna Nilsson,
Jonatan Rydberg, Dündar Göç, Susanne Schilliger Kildal, Gustav Hansson,
Fredrik Johnsson, Josefin Lövmark, Jakob Friman} och \emph{Christoffer
Johansson} för deras hjälpande händer.

\item Versionshanteringstjänsterna \emph{Bitbucket} och \emph{Mercurial} utan vilka
arbetet skulle befinna sig i fullständigt kaos.
\end{list}

Tack även till stöttande vänner och givmilda vädergudar.

%\section*{Bildrättigheter}
%Figur \ref{fig:knn-overview}: \copyright Antti Ajanki, 2007.\\

    
%I denna skall det anges vilka delar som skall tillskrivas respektive
%författare. Där skall också anges att en loggbok förts över de
%enskilda medverkandes prestationer under arbetet.
%\marginpar{+ kommentar om att vi kanske har vävt ihop våra bitar med varandra
%sedan\ldots (vilket vi bör göra snyggt)}
%Enligt reglerna får examination av examensarbete med flera författare
%inte ske utan att loggbok\footnote{Med loggbok menas här gruppens
%dagbok och sammanställning av de individuella tidsloggarna} inlämnats
%och rapportens inledning uppfyller villkoren ovan.

%\section*{Prestationsredovisning}
%Vi har bla bla bla
%
%Harald Freij
%
%Viktor Nilsson har skrivit den ursprunliga texten i följande kapitel:
%\ref{sec:knn}, \ref{sec:inledning}
%
%David Samuelsson
%
%Simon Sigurdhsson \ref{sec:features}, \ref{sec:resultat_features},
%\ref{sec:inledning}, \ref{sec:klassificering:fargrymd},
%\ref{sec:klassificering:ycbcr}, \ref{sec:klassificering:morfologi}, %\ref{sec:klassificering}

\end{document} 
%TODO: mer todo
