\documentclass[../rapport_MVEX01-11-05]{subfiles}
\begin{document}
	a
Moderna och innovativa gränssnitt för interaktion med datorer har länge varit
en synlig vision inom filmindustrin. Redan på mitten av nittiotalet hänfördes
tittare av 3d-gränssnitt i filmen Hackers, och i början av 2000-talet visade
scener i Minority Report hur Tom Cruise styr ett ett operativsystem med endast
sina händer, med hjälp av två trådlösa handskar.

Denna framtidsvision kan
ligga mycket närmre än vad Spielberg förväntat sig. Redan i början av
nittiotalet gjordes försök att tolka mänskliga rörelser i bildsekvenser
\cite{Yamato92}, och sedan dess har intresset i forskarvärlden bara ökat.
Handgester för kommunikation med datorer är alltså inte längre något som
hör hemma i filmens värld, utan kan snart vara en verklighet.

Det finns två angreppsmetoder till detta problem; den mer fysiskt krävande
metoden med sensorfyllda handskar och den mer lättillgängliga visionsbaserade
metoden som använder data från exempelvis webbkameror. Den senare har
fördelen att gemene man kan utnyttja lösningen utan några extra hårdvarukrav.

För den visionsbaserade tankegången är det mest grundläggande problemet
är att över huvud taget kunna identifiera en människas hud i bilder, och det
problemet har många lösningar. Metoderna förfinas hela tiden, men i stort har
det inte hänt något revolutionerande sedan början av 2000-talet
\cite{Sebe04,Kruppa02,Albiol01,Brand00}, med några få undantag
\citeaffixed{Hassanpour08,Khan10}{t.ex.}. När man väl löst detta problem måste datorn
även kunna tolka handens form och rörelser. Detta är ett svårare problem, men
har också behandlats flitigt \cite{Pavlovic97,Garg09,Nielsen04,Zabulis09}.

Trots denna uppmärksamhet från forskarvärden och även den kommersiella sektorn
på senare tid (se t.ex.~Kinect) så fortsätter gestigenkänning vara något som
kräver extern, ofta dyr hårdvara eller mycket kontrollerade miljöer för att
fungera bra. Detta problem kan endast lösas genom att implementera ett
videobaserat system som är tillräckligt robust för att kunna identifiera
handgester i mycket varierande miljöer.

Denna rapport behandlar både hudigenkänning och gestigenkänning, och ger en
robust lösning på gestigenkänningsproblemet i dess enklaste form, när
datorn endast förväntas känna igen statiska gester. Detta görs med hjälp av
Gaussian Mixture Models för hudklassificering och \knn-metoden för tolkning
av handgester utifrån en stor mängd egenskaper som extraheras ur bilddata.
Vidare lägger den en
teoretisk grund för att även kunna känna igen rörliga gester med hjälp av
så kallade gömda Markovmodeller.

\end{document} 
%TODO: mer todo
