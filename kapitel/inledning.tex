\documentclass[../rapport_MVEX01-11-05]{subfiles}
\begin{document}
\section{Inledning}
Var noggranna med att ange källor till det ni skriver. Jag
rekommenderar Vancouver-systemet\footnote{Även annat system accepteras
om det används konsekvent.} som är mest använt på MV. Man kan antingen
använda siffror [1], [2] etc, eller initialer som associerar till
författarnamnet(n) t.ex [BN], [BS] etc. Det senare kan vara lite
jobbigt om man har många källor men praktiskt om man har några som
huvudreferenser.

Läs mer i Fackspråks skrift: Utformning av rapporter och
kandidatarbetens skriftliga .... (2008-01-11)\cite{rapp}.
I det fall arbetet i huvudsak bygger på en eller ett par källor och
det är svårt att identifiera exakt när man använder respektive källa,
kan man tala om detta i inledningen. Man kan sedan referera till
källan om man återger en definition, en sats eller ett bevis eller på
annat sätt ligger nära källan. En direkt översättning kan jämställas
med ett citat, återberätta därför som om ursprunget var en skrift på
svenska så att ni håller er långt ifrån gränsen för plagiering. Är ni
osäkra på något så fråga examinator eller handledare.\footnote{Läs mer
om akademisk integritet på Chalmers webbsida \hfill \\ URL:
https://student.gate.chalmers.se/sv/regler$\_$rattigheter}

\notes{
\begin{itemize}
  \item Tempus: presens! \emph{Rapporten} i fokus, vad
  \emph{rapporten} vill åstadkomma, inte projektet
  \item Var \emph{tydligare}
  \item Se upp med bestämda former
  \item Mer CARS-tänkande (tratten)
  \item Introducera referenser!
  \item Anknyt mer till det som kommer i teorin!
\end{itemize}

}

\end{document} 
